\documentclass[pdf]{beamer}

\usepackage[utf8]{inputenc}
\usepackage[spanish]{babel}

\usepackage{ragged2e} % Paquete para trabajar texto.

\mode<presentation>

\usepackage{lmodern}
\usetheme{Madrid} 

\usepackage{float} %para fijar tablas
\usepackage{multirow} %Para tablas
\usepackage{booktabs}

\usepackage{csvsimple} %csv to table

\usepackage{hyperref}%Links
\hypersetup{  %formato link
    colorlinks=true,
    linkcolor=green,
    filecolor=blue,      
    urlcolor=blue}

% Para evitar errores de compilación en \author
\pdfstringdefDisableCommands{%
  \def\\{}%
  \def\vspace{}%
}

\usepackage{wrapfig} %para acomodar gráficos con texto.
\usepackage{graphicx} % para combinar gráficos con listas
\usepackage{graphbox}   % allows to add keys to \includegraphics
\usepackage{lipsum}     % only for testing purposes

\title[Curso Data Science]{\textbf{Análisis socioeducativo de los habitantes de la Ciudad de Buenos Aires}}

\author[Coderhouse]{\textbf{Profesor:} Damian Dapueto \\ \vspace{0.1cm} \textbf{Tutor:} Héctor Alonso \\ \vspace{0.1cm} \textbf{Grupo de Trabajo:} Lucia Buzzeo, Lucia Hukovsky,\\ Jose Saint German, Juan Martín Carini}

\date{}

\begin{document}
\justifying{ 

\begin{frame}

    \begin{center}
        \includegraphics[scale=0.2]{../Informe/Imagenes/Coder2.jpg}
    \end{center}

    \titlepage%

\end{frame}

\section{Introducción}

    \subsection{Resumen del proyecto}

\begin{frame}{Introdución}

\textbf{Principales hitos:} 
    \begin{itemize}
        \justifying{
        \item En la Ciudad Autónoma de Buenos Aires, se ha encontrado una gran limitación relacionada con el acceso equitativo a la educación.
        \item Para trabajar esta problemática, se ha recurrido a la \href{https://data.buenosaires.gob.ar/dataset/encuesta-anual-hogares/resource/3a45c563-396d-42de-ba93-8a93729e0723}{Encuesta Anual de Hogares} del Gobierno de la Ciudad de Buenos Aires para el año 2019. 
        \item Esta encuesta contiene información demográfica, social, económica, educativa y de salud de 14319 habitantes de la Ciudad.
        } 
    \end{itemize}

   \begin{figure}
       \includegraphics[scale=0.3]{../Informe/Imagenes/GobCiuBsAs.jpg}
    \end{figure}

\end{frame}

    \subsection{Objetivos del proyecto}

\begin{frame}{Objetivos}

    \begin{itemize}
        \justifying%
        \item Descubrir las principales variables intervinientes en el nivel máximo educativo alcanzado por la población de la Ciudad Autónoma de Buenos Aires (CABA).
        
        \item Predecir nuestra variable target ``Nivel Máximo Educativo’’ mediante dos modelos de clasificación:
        \begin{itemize}
            \justifying%
            \item \textbf{Árbol de decisión:} que construye un árbol durante el entrenamiento aplicado a la hora de realizar la predicción.
            \item \textbf{Bosque Aleatorio:} que es un conjunto (ensemble) de árboles de decisión combinados con bagging.
        \end{itemize}
    \end{itemize}
\end{frame}
\section{Planificación}
    \subsection*{Estructura de los trabajos}
\begin{frame}{Estructura de los trabajos}
    Este trabajo se ha dividido en 3 partes:
    \begin{enumerate}
        \justifying%
        \item \textbf{Introducción a las variables del problema:} Análisis exploratorio del dataset. 
        \item \textbf{Modelos analíticos:} Entrenamiento de modelos analíticos de clasificación.
        \item \textbf{Conclusión:} Conclusiones finales sobre los hallazgos, discusión de posibles limitaciones y futuras líneas de análisis.
        
    \end{enumerate}
\end{frame}

\section{Introducción a las variables: Análisis exploratorio de los datos}

\begin{frame}{Análisis exploratorio de los datos (EDA)}

    \textbf{Introducción a las variables:}
    \begin{itemize} 
        \item 31 variables
        \begin{itemize} 
            \item 10 numéricas 
            \item 21 categóricas. 
        \end{itemize}
    \end{itemize}
    Estas variables describen las siguientes características de los encuestados:
    \begin{itemize} 
        \item El nivel de ingresos
        \item El Sector educativo
        \item Los Factores geográficos
        \item Características de la Salud
        \item Descripción del grupo familiar
    \end{itemize}
    \vspace{4pt}

    \textbf{Identificación del Target:} Seleccionamos nuestra variable Target u objetivo para trabajar. Para reducir su dimensionalidad, la transformamos dejándola con cuatro valores:  \textbf{Inicial}, \textbf{Prim. Completo}, \textbf{Sec. Completo} y \textbf{Superior}.

\end{frame}
 
\subsection{Análisis univariado}
    \subsubsection{Género y edad}
\begin{frame}{Análisis univariado}
    \textbf{Análisis del género y la edad de los encuestados:}  
    \begin{center}
        \includegraphics[scale=0.315]{../Informe/Imagenes/AUGenero.png}
    \end{center}
    \begin{itemize}
        \item Género: categorías balanceadas
        \item Edad: Distribución normal
    \end{itemize}
\end{frame}
              
           
    \subsubsection{Máximo nivel educativo (Target)}
    
\begin{frame}{Análisis univariado}
    \textbf{Nivel máximo educativo:}     
    \begin{itemize}
        \item Moda de la variable: secundario completo
        \item El nivel secundario y primario explican casi el 77\% de los datos.
    \end{itemize}    
    \begin{center}
        \includegraphics[scale=0.225]{../Informe/Imagenes/AUTarget.png}    
    \end{center}
\end{frame}
    
    \subsection{Análisis bivariado}

\begin{frame}{Análisis bivariado}
    \vspace{-20pt}
    \textbf{Variable numéricas:}
    \vspace{4pt}

    Realizamos un mapa de calor para ver la interacción entre las variables numéricas.

    \begin{minipage}{0.5\textwidth}
        \vspace{10pt}
        \includegraphics[scale=0.22]{../Informe/Imagenes/ABSpearman.png}
    \end{minipage}
    \begin{minipage}{0.48\textwidth}
        \vspace{4pt}
        \begin{itemize}
            \justifying%
            \item No se observan fuertes correlaciones.
            \item ``años\_escolaridad'' correlaciona moderadamente bien con variables relacionadas al ingreso.
        \end{itemize}
    \end{minipage}

\end{frame}
 
    \subsubsection{Comparación entre variables numéricas}

\begin{frame}{Análisis bivariado}
    \textbf{Ingresos por edad:}
    \begin{center}
        \includegraphics[scale=0.15]{../Informe/Imagenes/ABCompVarNum.png}
    \end{center}
    Desde los 30 años en adelante el ingreso total de la persona se corresponde con el ingreso familiar. Por ende, suele haber un único ingreso fuerte por grupo familiar.
\end{frame}
 
    \subsubsection{Comparación de variables categóricas con numéricas}
\begin{frame}{Análisis bivariado}
    \textbf{Variable numéricas y categóricas}
    \vspace{6pt}

    Comparamos algunas variables con nuestro target, comenzando con los ingresos totales.
    \begin{figure}
        \includegraphics[
                         width=12cm,
                         height=4cm
                         ]{../Informe/Imagenes/ABCatVsNum3Bis.png}
    \end{figure}

\end{frame}

\begin{frame}{Análisis bivariado}

    Distribución de los ingresos familiares con respecto a nuestro target:
    \begin{center}
        \includegraphics[scale=0.21]{../Informe/Imagenes/ABCatVsNum5.png}
    \end{center}
    En definitiva, se observa un desplazamiento de los valores centrales (dentro de la caja) hacia la izquierda a medida que aumenta el nivel educativo.
\end{frame}
    \subsubsection{Variable numéricas con comuna}
\begin{frame}{Análisis bivariado}
    \begin{minipage}{0.55\textwidth}
        \vspace{10pt}
        \includegraphics[scale=0.28]{../Informe/Imagenes/ABCatVsNum6.png}
    \end{minipage}
    \begin{minipage}{0.4\textwidth}
        \begin{itemize}
            \vspace{4pt}
            \justifying%
            \item Sur de la ciudad: mayor cantidad de encuestados con nivel inicial, primario y secundario completo,
            \item Norte (particularmente el barrio de Palermo): mayor cantidad de personas con estudios superiores,
            \item Comunas del este (``centro de la ciudad''): Alta cantidad de encuestados con nivel superior.
        \end{itemize}
    \end{minipage}

\end{frame}    

    \subsection{Análisis multivariado}

\begin{frame}{Análisis multivariado}
    \footnotesize
    Probamos de cruzar años de escolaridad, nivel máximo educativo y los ingresos totales.
    \vspace{10pt}
    \begin{minipage}{0.5\textwidth}
        \includegraphics[scale=0.3]{../Informe/Imagenes/AMIngVsAnosEsc.png}
    \end{minipage}
\begin{minipage}{0.45\textwidth}
    \begin{itemize}
        \footnotesize
        \justifying%
        \item Hasta los 6 años todos los casos llegan al nivel inicial.
        \item Vemos dos años en que aparece el primario completo: 7 y 12 años. Estimamos que se debe a la división entre los que comenzaron su educación en la primaria y los que comenzaron en el nivel inicial.
        \item A partir de los 12 años: un aumento consistente de los ingresos totales.
    \end{itemize}
\end{minipage}
\end{frame} 
\begin{frame}{Análisis multivariado}
    \vspace{-5pt}
    \begin{figure}[H]
        \includegraphics[width=\textwidth,
                         height=6cm]{../Informe/Imagenes/AMIngVSTargetBis.png}
    \end{figure}
    \vspace{-10pt}
    
    
    Podemos ver que los casos que no provienen de villas de emergencia obtienen en promedio \textbf{ingresos más altos en todos los niveles educativos}. El alcanzar estudios superiores no parece homogeneizar ambos conjuntos. 
\end{frame}

\begin{frame}{Análisis multivariado}
    Ingresos familiares según el máximo nivel educativo alcanzado:

    \begin{minipage}{0.55\textwidth}
        \vspace{10pt}
        \begin{figure} 
        \includegraphics[scale=0.26]{../Informe/Imagenes/AMIngVsComuna.png}
        \end{figure}
    \end{minipage}
    \begin{minipage}{0.38\textwidth}
        \begin{itemize}
            \justifying%
            \item a mayor nivel educativo, menor varianza de los ingresos familiares entre comunas
            \item ¿incluir a menores de edad sesga los valores? Queda pendiente realizar ese análisis
        \end{itemize}
    \end{minipage}

\end{frame}

\section{Modelos analíticos}

\begin{frame}{Modelos analíticos}

    \begin{Large}
        \textbf{Transformación de variables}
    \end{Large}
    \vspace{15pt}

    Transformamos variables para poder trabajar con los algoritmos:
    \begin{itemize}
        \item Recategorización de la variable ``Target'' en variables numéricas:
        \begin{itemize}
            \item \textbf{inicial=} 1,
            \item \textbf{prim\_completo=} 2,
            \item \textbf{sec\_completo=} 3,
            \item \textbf{superior=} 4,
        \end{itemize}
        \item Reagrupación la variable ``comuna'' por regiones para reducir la dimensionalidad (norte, centro, sur y oeste),
        \item Eliminamos algunas variables que no resultan relevantes 
    \end{itemize}
    
\end{frame}

    \subsection{Tratados de nulos}
    
\begin{frame}{Modelos analíticos}
    \begin{Large}
        \textbf{Tratados de nulos}
    \end{Large}
    \vspace{4pt}

    Encontramos valores nulos en estas variables, por lo que para trabajar con el modelado realizamos las siguientes acciones:
    \begin{table}[H]
        \begin{tabular}{lrl}
            \toprule
            \textbf{Variable}                & \textbf{Nulos} & \textbf{Acción} \\ \midrule
            situacion\_conyugal     & 1     & Reemplazamos con la moda\\ 
            lugar\_nacimiento       & 1     & Reemplazamos con la moda\\ 
            sector\_educativo       & 3     & Reemplazamos con la moda\\ 
            afiliacion\_salud       & 4     & Reemplazamos con la moda\\ 
            años\_escolaridad       & 62    & Reemplazamos con la \\
                                    &       & mediana por comuna y sexo\\ 
            nivel\_max\_educativo   & 1054  & Eliminamos la variable\\ 
            Target                  & 1096  & Eliminamos sus nulos y transformamos\\
                                    &       & su tipo a entero\\ 
            hijos\_nacidos\_vivos   & 7784  & Reemplazamos con la moda\\ 
            \bottomrule
        \end{tabular}
    \end{table}
    
\end{frame}

    \subsection{Target}

        \subsubsection{Borrado de variables}
    
    \subsection{Procesamiento}

    \subsection{Árbol de decisión}
        
\begin{frame}{Modelos analíticos}
    \begin{Large}
        \textbf{Árbol de decisión}
    \end{Large}
    \vspace{10pt}

     \textbf{1. Primer modelo:}

    Parámetros:
    \begin{itemize}
        \item max\_depth=8,
        \item criterion='gini',
    \end{itemize}
    
    El Accuracy score para el test es de: \textbf{0.940} y las métricas:
    \begin{table}[H]
        \scriptsize
        \centering
        \begin{tabular}{rcccc}
            \toprule
             & precision & recall & f1-score & support \\ \midrule
            Inicial    & 1.00& 1.00 & 1.0 & 446 \\ 
            Primario   & 0.93 & 0.99 & 0.95 & 978 \\ 
            Secundario & 0.95 & 0.96 & 0.95 & 1771 \\ 
            Superior   & 0.90 & 0.82 & 0.86 & 772 \\ 
            & & & & \\
            accuracy & & & 0.94 & 3967 \\ 
            macro avg & 0.94 & 0.94 & 0.94 & 3967 \\ 
            weighted avg & 0.94 & 0.94 & 0.94 & 3967 \\ 
            \bottomrule
        \end{tabular}
    \end{table}
\end{frame}
\begin{frame}{Modelos analíticos}
    \begin{Large}
        \textbf{Resultados}
    \end{Large}
    \vspace{10pt}

     \begin{itemize}
        \item \textbf{Bias o sesgo:} 96.89\% $\Rightarrow$ poco error $\Rightarrow$ sesgo bajo,
        \item \textbf{Variance=Test\_Score $-$ Bias=} 2.89\% $\Rightarrow$ varianza baja.        
     \end{itemize}
    \vspace{6pt}
    
    Entonces, el modelo tiene una \textbf{buena relación} de sesgo y varianza.
    \vspace{10pt}
    
    Sin embargo, tenemos que la variable ``años\_escolaridad'' tiene una importancia del 84\%, por mucho superior al resto de variables.
    \vspace{10pt}

    Por lo tanto, desarrollamos un nuevo modelo sin esta variable. 
\end{frame}
        \subsubsection{Segundo modelo}
\begin{frame}{Modelos analíticos}
    \textbf{2. Segundo modelo:}
    \vspace{10pt}
    
    Esta vez al correr el modelo, utilizaremos el ``DecisionTreeClassifier'' sin la variable años\_escolaridad y con el criterion entropy.
    \begin{table}[!ht]
        \scriptsize
        \centering
        \begin{tabular}{rcccc}
            \toprule
             & precision & recall & f1-score & suppo \\ \midrule
            Inicial    & 0.83 & 0.79 & 0.81 & 446 \\
            Primario   & 0.45 & 0.26 & 0.33 & 978 \\
            Secundario & 0.56 & 0.66 & 0.60 & 1771 \\
            Superior   & 0.39 & 0.45 & 0.42 & 772 \\
            & & & & \\
            accuracy & & & 0.53 & 3967 \\
            macro avg & 0.56 & 0.59 & 0.54 & 3967 \\
            weighted avg & 0.53 & 0.53 & 0.52 & 3967 \\
            \bottomrule
        \end{tabular}
    \end{table}
\end{frame}

\begin{frame}{Modelos analíticos}
    \begin{Large}
        \textbf{Resultados}
    \end{Large}
    \vspace{10pt}

    \begin{itemize}
        \item \textbf{Bias o sesgo:} 99.78\% $\Rightarrow$ poco error $\Rightarrow$ sesgo bajo,
        \item \textbf{Variance=Test\_Score $-$ Bias=} 46.39\% $\Rightarrow$ varianza bastante alta,
    \end{itemize}
    Lo que nos da como resultado, que este modelo esta haciendo \textbf{OVERFITTING}.
\vspace{10pt}
    Por lo que se observa, el árbol performa bastante peor sin esta variable, aumentando especialmente la varianza. Por lo tanto optamos probar mejorar nuestro modelo con un grid search.
\end{frame}

    \subsubsection{Gridsearch con CV}

\begin{frame}{Modelos analíticos}
    \begin{Large}
        \textbf{Gridsearch con CV}
    \end{Large}
    \vspace{10pt}
    
    En la grilla de parámetros para el Gridsearch elegimos los siguientes:
    \begin{itemize}
        \item Profundidad máxima del árbol: rango entre 5 y 10 niveles,
        \item Cantidad máxima de features: rango entre 11 y 13,
        \item Usamos todos los criterios posibles para el split: gini, entropy y log\_loss;
    \end{itemize}
    \vspace{10pt}

    A su vez, realizamos un cross validation partiendo el dataframe en 10 secciones.
    \vspace{10pt}

    Resultado: el mejor árbol de decisión posible obtiene 0.642, con las siguientes características.
    
    \begin{itemize}
        \item Profundidad de  6
        \item Utilizar  10  variables 
        \item Usar el método ``gini''
    \end{itemize}
\end{frame}
\begin{frame}{Modelos analíticos}
    
    \begin{Large}
        \textbf{3.Tercer Modelo}
    \end{Large}
    \vspace{10pt}

    Entonces, entrenamos el modelo bajo estos mismos parámetros y obtenemos el siguiente reporte de clasificación:
    \begin{table}[H]
        \scriptsize
        \centering
        \begin{tabular}{rcccc}
            \toprule
             & precision & recall & f1-score & suppo \\ \midrule
            Inicial    & 0.99 & 0.74 & 0.85 & 446 \\
            Primario   & 0.49 & 0.21 & 0.30 & 978 \\
            Secundario & 0.55 & 0.87 & 0.69 & 1771 \\
            Superior   & 0.78 & 0.41 & 0.54 & 772 \\
            accuracy & & & 0.60 & 3967 \\
            macro avg & 0.70 & 0.56 & 0.59 & 3967 \\
            weighted avg & 0.63 & 0.60 & 0.57 & 3967 \\
            \bottomrule
        \end{tabular}
    \end{table}

\end{frame}

\begin{frame}{Modelos analíticos}

    \begin{Large}
        \textbf{Resultados}
    \end{Large}
    \vspace{10pt}    

    \begin{itemize}
        \item \textbf{Bias o sesgo:} 65.27\% $\Rightarrow$ bastantes errores $\Rightarrow$ high bias,
        \item \textbf{Variance=Test\_Score $-$ Bias:} 5.14\%  $\Rightarrow$ low variance.
    \end{itemize}
    
    \vspace{6pt} 
    Por lo tanto, el modelo esta haciendo \textbf{UNDERFITTING}
    \vspace{6pt}    

    \textbf{Conclusiones de mejora de modelo:}
    \begin{itemize}
        \item Varianza: de 44.97\% a 5.14\% $\Rightarrow$ \textbf{MEJORÓ}
        \item Accuracy: de 100\% a 65.27\%  $\Rightarrow$ \textbf{EMPEORÓ}
    \end{itemize}

\end{frame}

    \subsection{Random Forest Classifier}
\begin{frame}{Modelos analíticos}
    
    \begin{Large}
        \textbf{Random Forest Classifier}
    \end{Large}
    \vspace{10pt}
    
    \textbf{4. Cuarto modelo:}
    
    \vspace{6pt}
    En esta instancia volvemos a incluir la variable años de escolaridad.
    
    \vspace{6pt}
    En nuestro tercer modelo utilizamos el Random Forest Classifier con los siguientes parámetros:
    \begin{itemize}
        \item n\_estimators (cantidad de árboles de decisión generados)=200,
        \item max\_depth=15,
        \item criterion='gini.
    \end{itemize}

\end{frame}

\begin{frame}{Modelos analíticos}

    \begin{Large}
        \textbf{Random Forest Classifier}
    \end{Large}
    \vspace{10pt}
    
    Que nos da los siguientes resultados en cuanto a las métricas:
    \begin{table}[H]
        \scriptsize
        \centering
        \begin{tabular}{rcccc}
            \toprule
                & precision & recall & f1-score & suppo \\ \midrule
            Inicial    & 1.00 & 0.96 & 0.98 & 446 \\
            Primario   & 0.86 & 0.95 & 0.90 & 978 \\
            Secundario & 0.91 & 0.93 & 0.9 & 1771 \\
            Superior   & 0.92 & 0.76 & 0.83 & 772 \\
            & & & & \\
            accuracy & & & 0.91 & 3967 \\
            macro avg & 0.92 & 0.90 & 0.91 & 3967 \\
            weighted avg & 0.91 & 0.91 & 0.90 & 3967 \\
            \bottomrule
        \end{tabular}
    \end{table}

    El random forest performa bastante bien, es decir, mucho mejor que los modelos anteriores.

\end{frame}

\begin{frame}{Modelos analíticos}

    \begin{Large}
        \textbf{Resultados}
    \end{Large}
    \vspace{10pt}    
    
    \begin{itemize}
        \item \textbf{Bias o sesgo:} 97.80\% $\Rightarrow$ pocos errores $\Rightarrow$ sesgo bajo,
        \item \textbf{Variance=Test\_Score $-$ Bias:} 7.20\% $\Rightarrow$ la varianza es baja.
    \end{itemize}
    
    Obtuvimos un buen modelo. No obstante, buscamos cuales son las variables más importantes. Encontramos que los años de escolaridad redujo la enorme importancia (a un 43.58\%) que tenía en el random tree. 
    
    \vspace{10pt}
    
    Sin embargo, sigue correspondiendo quitarla del modelo.

\end{frame}

\subsubsection{Cuarto modelo}

\begin{frame}{Modelos analíticos}

    \textbf{5. Quinto modelo:}
    \vspace{10pt}

    En este caso elegimos los siguientes parámetros, quitando la variable años de escolaridad:
    \begin{itemize}
        \item n\_estimators=200,
        \item max\_depth=10,
        \item criterion='gini'.
    \end{itemize}
    
    Dándonos por resultado los siguientes medidas de desempeño:
    
    \begin{table}[H]
        \scriptsize
        \centering
        \begin{tabular}{rcccc}
            \toprule
             & precision & recall & f1-score & suppo \\ \midrule
            Inicial    & 0.95 & 0.78 & 0.86 & 446 \\
            Primario   & 0.54 & 0.23 & 0.32 & 978 \\
            Secundario & 0.56 & 0.88 & 0.69 & 1771 \\
            Superior   & 0.80 & 0.40 & 0.53 & 772 \\
            & & & & \\
            accuracy & & & 0.62 & 3967 \\
            macro avg & 0.71 & 0.57 & 0.60 & 3967 \\
            weighted avg & 0.65 & 0.62 & 0.59 & 3967 \\
            \bottomrule
        \end{tabular}
    \end{table}

\end{frame}

\begin{frame}{Modelos analíticos}

    \begin{Large}
        \textbf{Resultados}
    \end{Large}
    \vspace{10pt}

    \begin{itemize}
        \item \textbf{Bias o sesgo:} 89.11\% $\Rightarrow$ pocos errores $\Rightarrow$ sesgo bajo,
        \item \textbf{Variance=Test\_Score $-$ Bias=} 27.4\% $\Rightarrow$ varianza alta.
    \end{itemize}
    
    El modelo empeora su accuracy pero está muy cercano al mejor modelo de Random Tree, mientras que crece mucho la varianza. Vamos a probar mejorándolo con grid search.

\end{frame}

    \subsubsection{Gridsearch con CV}

\begin{frame}{Modelos analíticos}

    \begin{Large}
        \textbf{Gridsearch con CV}
    \end{Large}
    \vspace{10pt}

    En la grilla de parámetros para el Gridsearch elegimos los siguientes:
    \begin{itemize}
        \item Profundidad máxima: 5,7,10,15 de profundidad, agregando la opción de que no tenga máximo,
        \item Cantidad máxima de features: 5,8,10,30,41,
        \item Número de estimadores: 200,300,500,
        \item Criterion: [`gini',`entropy',`log\_loss'],
        \item También realizamos un cross validation de 10 particiones.
    \end{itemize}
    
    Como resultado, el mejor random forest posible obtiene 0.668. Para eso el árbol debe tener: 
    \begin{itemize}
        \item una profundidad de 15, 
        \item utilizar  10  variables, 
        \item tener  300  estimadores 
        \item y utilizar el método ``gini''.
    \end{itemize}

\end{frame}

\begin{frame}{Modelos analíticos}    

    \begin{Large}
        \textbf{6. Sexto Modelo}
    \end{Large}
    \vspace{10pt}

    Entonces, entrenamos el modelo bajo estos mismos parámetros y obtenemos el siguiente reporte de clasificación:
    
    \begin{table}[!ht]
        \scriptsize
        \centering
        \begin{tabular}{rcccc}
            \toprule
             & precision & recall & f1-score & support \\ \midrule
            Inicial    & 0.95 & 0.79 & 0.86 & 446 \\
            Primario   & 0.56 & 0.23 & 0.33 & 978 \\
            Secundario & 0.56 & 0.89 & 0.69 & 1771 \\
            Superior   & 0.80 & 0.40 & 0.54 & 772 \\
            & & & & \\
            accuracy & & & 0.62 & 3967 \\
            macro avg & 0.72 & 0.58 & 0.60 & 3967 \\
            weighted avg & 0.65 & 0.62 & 0.59 & 3967 \\
            \bottomrule
        \end{tabular}
    \end{table}

\end{frame}

\begin{frame}{Modelos analíticos}

    \begin{Large}
        \textbf{Resultados}
    \end{Large}
    \vspace{10pt}

    \begin{itemize}
        \item \textbf{Bias o sesgo:} 90.65\% $\Rightarrow$ pocos errores $\Rightarrow$ sesgo bajo,
        \item \textbf{Variance=Test\_Score $-$ Bias:} 28.54\% $\Rightarrow$ varianza alta. 
    \end{itemize} 

    Lo que nos indica que nuestro modelo esta haciendo \textbf{OVERFITING}.
    Al utilizar random forest  hemos podido mejorar el sesgo y disminuir el underfitting en 2 puntos porcentuales aproximadamente.
    \vspace{10pt}
    
    Sin embargo, se vio afectada la varianza en estos modelos, que pasó de estar alrededor del 5\% en el árbol de decisión mejorado a 28\%.

\end{frame}

\section{Conclusiones Finales}

\begin{frame}{Conclusiones Finales}
    
    Finalmente, tomamos las métricas de cada uno de ellos y hacemos un cuadro comparativo:
    
    \begin{table}[H]
        \scriptsize
        \centering
        \begin{tabular}{ lccccccc }
            \toprule
            Modelo & accuracy & sesgo & varianza & f1\_inicial & f1\_pri & f1\_sec & f1\_sup \\ \midrule
            2. Árbol\_default & 0.53 & 1.00 & 0.46 & 0.81 & 0.33 & 0.60 & 0.42 \\  
            3. Árbol\_mejorado & 0.60 & 0.65 & 0.05 & 0.85 & 0.30 & 0.69 & 0.53 \\ 
            5. Bosque\_default & 0.62 & 0.89 & 0.27 & 0.86 & 0.32 & 0.69 & 0.53 \\ 
            6. Bosque\_mejorado & 0.62 & 0.91 & 0.29 & 0.86 & 0.33 & 0.69 & 0.54 \\
            \bottomrule
        \end{tabular}
    \end{table}

    \begin{itemize}
        \item El árbol default tiene el mejor resultado con respecto al sesgo, pero su varianza lo deja afuera de la competencia.
        \item Por el contrario, el árbol mejorado tiene una varianza insuperable de 5\%, aunque con el menor puntaje con respecto al sesgo.
        \item El bosque default tiene resultados mixtos en ambas categorías.
        \item El bosque mejorado destaca por bajo sesgo pero su varianza es la segunda peor.
    \end{itemize}
    
\end{frame}

\begin{frame}{Conclusiones Finales}

    Los finalistas son \textbf{el árbol y el bosque mejorado}. Ambos performan muy bien pero en métricas diferentes. 
    \vspace{10pt} 

    En nuestra opinión, es \textbf{el árbol mejorado el ganador}, ya que:
    \begin{itemize}
        \item Tiene la robustez suficiente para poder generalizar en caso de agregar nuevos datos al modelo. 
        \item Tiene mayor velocidad de entrenamiento 
        \item Tiene mayor capacidad de ser visualizada en un gráfico.
    \end{itemize}
    \vspace{10pt} 

    En futuras líneas de investigación se debería investigar en profundidad el desbalanceo de datos propio del Target y mitigar la problemática agregando datos en categorías con deficit y eliminando datos de categorías en exceso.

\end{frame}
}

\end{document}