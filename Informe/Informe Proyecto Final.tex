\documentclass[a4paper]{article}

\usepackage[utf8]{inputenc}
\usepackage[spanish]{babel}

\usepackage[top=2cm, bottom=1.5cm, right=1.5cm, left=2cm]{geometry}

\usepackage{graphicx}

\usepackage{hyperref}%Links
\hypersetup{  %formato link
    colorlinks=true,
    linkcolor=blue,
    filecolor=blue,      
    urlcolor=blue}

\usepackage{float} %para fijar tablas
\usepackage{multirow} %Para tablas

\usepackage{csvsimple} %csv to table

\title{\textbf{CoderHouse \\ \vspace{0.5cm} Curso Data Science \\ \vspace{0.5cm} Informe del Proyecto Final\\ \vspace{0.5cm}
Análisis socioeducativo de los habitantes de la Ciudad de Buenos Aires}}

\author{\textbf{Profesor:} Damian Dapueto \\ \vspace{0.2cm} \textbf{Tutor:} Héctor Alonso \\ \vspace{0.2cm} \textbf{Grupo de Trabajo:} \\ Lucia Buzzeo, Lucia Hukovsky,\\ Jose Saint German, Juan Martín Carini}

\begin{document}

\maketitle

\begin{center}
    \includegraphics[scale=1.75]{Imagenes/Coder.jpg}
\end{center}
 
\thispagestyle{empty}
 
\newpage
 
\tableofcontents
 
\newpage

\section{Introducción}

    \subsection{Resumen del proyecto}

        La ciudadanía es un concepto jurídico, filosófico y político que ha sido creado para designar a una persona física que constituye una sociedad o entidad territorial. Para las personas que forman parte de una comunidad, ciudadanos, resulta de suma importancia sentirse representados por los demás integrantes de la misma, mediante políticas públicas que abarquen sus necesidades y requerimientos. 

        La toma de datos demográficos y la estadística son dos herramientas primordiales a la hora de identificar requerimientos de los integrantes de una comunidad. Dichas herramientas describen, de forma cuantitativa, a la sociedad bajo estudio. Precisamente, los censos y la estadística son la fuente primaria de información para la planificación económica y social de una población, por parte de sus representantes.

        En el caso particular de Argentina, el Instituto Nacional de Estadística y Censos (INDEC) es el organismo público que ejerce la dirección superior de todas las actividades estadísticas oficiales que se realizan en el país. La información que produce el INDEC es una herramienta básica para la planificación de políticas públicas, así como para las investigaciones y proyecciones que se realizan en los ámbitos académico y privado.

        Al adentrarse y estudiar los índices correspondientes a uno de los ejes principales, educación, en un territorio delimitado, Ciudad Autónoma de Buenos Aires, se ha encontrado una gran limitación relacionada con su acceso inequitativo para los diferentes actores de la sociedad. 

        Este hecho tiene consecuencias de índole social y económico para la población. Sin embargo, la principal problemática se da a nivel individual, y radica en el impedimento al acceso educativo para un porcentaje de la sociedad. Esto no ha resultado una novedad para el grupo, pero sí ha dado el pie a la búsqueda de una respuesta teórica a dicha disparidad, en concreto, a descubrir las principales variables que afectan el nivel educativo.

        El análisis realizado en el marco del presente proyecto podría establecer una base de requerimientos que permitan generar políticas públicas efectivas, no solo en el ámbito educativo, sino en el económico, cultural, social y geográfico, entre otros.

    \subsection{Definición de la fuente de información}

        Para trabajar esta problemática, se ha recurrido a la \href{https://data.buenosaires.gob.ar/dataset/encuesta-anual-hogares/resource/3a45c563-396d-42de-ba93-8a93729e0723}{Encuesta Anual de Hogares} del Gobierno de la Ciudad de Buenos Aires para el año 2019. El dataset está disponible en la base de datos abiertos del GCBA.

        Esta encuesta contiene información demográfica, social, económica, educativa y de salud de 14319 habitantes de la Ciudad, la cual es una muestra representativa que permite obtener un vistazo de la población de la Ciudad.

    \subsection{Objetivos del proyecto}

        Entre los objetivos del proyecto, se encuentra descubrir las principales variables intervinientes en el nivel máximo educativo alcanzado por la población de la Ciudad Autónoma de Buenos Aires (CABA).

        Pero como objetivo principal intentaremos generar un modelo de predicción aplicado a nuestra variable target ``Nivel Máximo Educativo’’, esto lo haremos implementando los siguientes modelos de clasificación:
        \begin{itemize}
            \item \textbf{Árbol de decisión:} que construye un árbol durante el entrenamiento que es el que aplica a la hora de realizar la predicción.
            \item \textbf{Bosque Aleatorio:} que es un conjunto (ensemble) de árboles de decisión combinados con bagging.
        \end{itemize}
        A partir de la obtención del mejor árbol de decisión y el mejor bosque aleatorio tomaremos la decisión de cuál de los dos es el mejor algoritmo para lograr los objetivos de este trabajo.


        %ORIGINAL

        % En el presente proyecto se persigue el objetivo de descubrir las principales variables intervinientes en el nivel máximo educativo alcanzado por la población de la Ciudad Autónoma de Buenos Aires (CABA).
        
        % En concreto, con este proyecto se plantea el análisis de variables que podrían contribuir al nivel máximo educativo alcanzado por cada individuo en CABA. Las mismas se estudian tanto de forma independiente como entrelazadas. El alcance de dicho objetivo permitiría identificar posibles causas que derivan en la desigualdad de acceso educativo, y, en última instancia, generar políticas públicas eficaces que permitan subsanar dicha problemática. 

        % Y por último intentaremos generar un modelo de predicción aplicado a nuestra variable target ``Nivel Máximo Educativo’’, esto lo haremos implementando los siguientes modelos de clasificación:
        % \begin{itemize}
        %     \item \textbf{Árbol de decisión:} que construye un árbol durante el entrenamiento que es el que aplica a la hora de realizar la predicción.
        %     \item \textbf{Bosque Aleatorio:} que es un conjunto (ensemble) de árboles de decisión combinados con bagging.
        % \end{itemize}
        % A partir de la obtención del mejor árbol de decisión y el mejor bosque aleatorio tomaremos la decisión de cuál de los dos es el mejor algoritmo para lograr los objetivos de este trabajo.
    
    % \subsection{Definición de la fuente de información}

    %     Para trabajar esta problemática, se ha recurrido a la \href{https://data.buenosaires.gob.ar/dataset/encuesta-anual-hogares/resource/3a45c563-396d-42de-ba93-8a93729e0723}{Encuesta Anual de Hogares} del Gobierno de la Ciudad de Buenos Aires para el año 2019. El dataset está disponible en la base de datos abiertos del GCBA.

    %     Esta encuesta contiene información demográfica, social, económica, educativa y de salud de 14319 habitantes de la Ciudad, la cual es una muestra representativa que permite obtener un vistazo de la población de la Ciudad.


\newpage

\section{Planificación}

    \subsection*{Estructura de los trabajos}

        Este trabajo se ha dividido en 3 partes:
        \begin{enumerate}
            \item \textbf{Introducción a las variables del problema:} Se realiza un análisis de las variables del dataset. En el mismo se busca conocer su performance dentro del dataset. A la vez, se investiga cómo las variables interactúan entre sí. Esta parte es lo que se conoce como análisis univariado, bivariado y multivariado.
            \item \textbf{Modelos analíticos:} En esta sección se entrenan diversos modelos analíticos y algoritmos que sirven para alcanzar los objetivos seteados para el presente proyecto. Como la variable objetivo es categórica, se realizan modelos de clasificación.
            \item \textbf{Conclusión:} Se alcanzan conclusiones finales sobre los hallazgos. Además, se discuten posibles limitaciones y se plantean futuras líneas de análisis, a partir del análisis presente.
            
        \end{enumerate}

\newpage

\section{Introducción a las variables: Análisis exploratorio de los datos}
 
    En el análisis explotarotio de los datos se ha buscado definir las variables que componen el dataset. En ese sentido, luego de cargar el dataset, se ha comenzado por conocer los tipos de datos y si existían nulls.
   
    \begin{table}[H]\begin{center}
    \begin{tabular}{clll}
    \multicolumn{4}{l}{Tamaño del set de datos: 14319 entadas, 0 a 14318} \\
    \multicolumn{4}{l}{Con un total 31 columnas:}  \\
    \#  & Columnas                   & Entradas No Vacías & Tipo de Dato \\ \hline
    0  & id                          & 14319 non-null & int64 \\
    1  & nhogar                      & 14319 non-null & int64 \\
    2  & miembro                     & 14319 non-null & int64 \\
    3  & comuna                      & 14319 non-null & int64 \\
    4  & dominio                     & 14319 non-null & object \\
    5  & edad                        & 14319 non-null & int64 \\
    6  & sexo                        & 14319 non-null & object \\
    7  & parentesco\_jefe             & 14319 non-null & object \\
    8  & situación\_conyugal          & 14318 non-null & object \\
    9  & num\_miembro\_padre           & 14319 non-null & object \\
    10 & num\_miembro\_madre           & 14319 non-null & object \\
    11 & estado\_ocupacional          & 14319 non-null & object \\
    12 & cat\_ocupacional             & 14319 non-null & object \\
    13 & calidad\_ingresos\_lab        & 14319 non-null & object \\
    14 & ingreso\_total\_lab           & 14319 non-null & int64  \\
    15 & calidad\_ingresos\_no\_lab     & 14319 non-null & object \\
    16 & ingreso\_total\_no\_lab        & 14319 non-null & int64  \\
    17 & calidad\_ingresos\_totales    & 14319 non-null & object \\
    18 & ingresos\_totales            & 14319 non-null & int64  \\
    19 & calidad\_ingresos\_familiares & 14319 non-null & object \\
    20 & ingresos\_familiares         & 14319 non-null & int64  \\
    21 & ingreso\_per\_capita\_familiar & 14319 non-null & int64  \\
    22 & estado\_educativo            & 14319 non-null & object \\
    23 & sector\_educativo            & 14316 non-null & object \\
    24 & nivel\_actual                & 14319 non-null & object \\
    25 & nivel\_max\_educativo         & 13265 non-null & object \\
    26 & años\_escolaridad            & 14257 non-null & object \\
    27 & lugar\_nacimiento            & 14318 non-null & object \\
    28 & afiliacion\_salud            & 14315 non-null & object \\
    29 & hijos\_nacidos\_vivos         & 6535 non-null  & object \\
    30 & cantidad\_hijos\_nac\_vivos    & 14319 non-null & object \\
    \multicolumn{4}{l}{Tipos de datos: int64(10), object(21)}  \\
    \multicolumn{4}{l}{Memoria usada: 3.4+ MB} \\
    \end{tabular}\end{center}
    \caption{Análisis preliminar de variables del dataset.}
    \label{Preliminar dataset analysis}
    \end{table}
       
    Se puede observar que se cuenta con 10 variables numéricas y 31 variables categóricas. 
    En base a los datos arrojados por la tabla de arriba, se han generado diversas transformaciones de variables, así como la creación de la variable ``Target'', pues es la que se usará para todo el análisis:
    \begin{itemize}
        \item Creación de la variable ``Target'' definida por la variable ``nivel\_max\_educativo''.
        \item En la variable ``Target'', se ha reducido su dimensionalidad intercambiando los valores:
            \begin{itemize}
                \item ``Secundario/medio común'' y ``EGB (1° a 9° año)'' por ``sec\_completo'',
                \item ``Primario especial'' y ``Primario común'' por ``prim\_completo'',
                \item ``Sala de 5'' por ``incial'',
                \item ``Otras escuelas especiales" por ``superior'',
                \item y por último a ``No corresponde'' por nulos.
            \end{itemize}
        \item Reemplazo de los valores de ``años\_escolaridad'' para que todos sean numéricos.
        \item En la variable ``cantidad\_hijos\_nac\_vivos'' se ha cambiado el valor ``no corresponde'' a nulo, para luego cambiar el tipo de variable a entero.
        \item Las variables ``comuna'', ``id'', ``nhogar'' y ``miembro'' son de tipo numérico, pero deberían ser categóricas, por lo tanto se ha transformado su tipo a string.
        \item Por último, se han renombrado algunas variables para mejorar la comprensión de su función:
            \begin{itemize}
                \item ``dominio\_Villas\_de\_emergencia'' por ``dominio\_villas'',
                \item ``ingreso\_per\_capita\_familiar" por ``ing\_per\_cap\_familiar'',
                \item ``cantidad\_hijos\_nac\_vivos'' por ``cant\_hijos\_nac\_vivos''.
            \end{itemize}
    \end{itemize}

    Por lo tanto, se detalla el diccionario de las variables actualizadas según los cambios indicados previamente:
    \begin{table}[H]
    \begin{tabular}{|l|l|}
        \hline
        \textbf{Variables}                     & \textbf{Descripción} \\ \hline
        \textbf{id}                            & Clave que identifica la vivienda \\ \hline
        \textbf{nhogar}                        & La variable id + nhogar = clave que identifica a cada hogar \\ \hline
        \textbf{miembro}                       & Variables id + nhogar + miembro = clave que identifica  cada persona \\ \hline
        \textbf{comuna}                        & Comuna donde reside la persona encuestada \\ \hline
        \textbf{edad}                          & Edad de la persona encuestada \\ \hline
        \textbf{sexo}                          & Sexo de la persona encuestada \\ \hline
        \textbf{parentesco\_jefe}              & Parentesco entre la persona encuestada y el jefe de hogar \\ \hline
        \textbf{situacion\_conyugal}           & Situación conyugal de la persona encuestada \\ \hline
        \textbf{num\_miembro\_padre}           & Número de miembro que corresponde al padre \\ \hline
        \textbf{num\_miembro\_madre}           & Número de miembro que corresponde a la madre \\ \hline
        \textbf{estado\_ocupacional}           & Situación ocupacional de la persona encuestada \\ \hline
        \textbf{cat\_ocupacional}              & Categoría ocupacional de la persona encuestada \\ \hline
        \textbf{calidad\_ingresos\_lab}        & Calidad de la declaración de ingresos laborales totales \\ \hline
        \textbf{ingreso\_total\_lab}           & Ingreso total laboral percibido el mes anterior \\ \hline
        \textbf{calidad\_ingresos\_no\_lab}    & Calidad de la declaración de ingresos no laborable totales \\ \hline
        \textbf{ingreso\_total\_no\_lab}       & Ingreso total no laboral percibido el mes anterior \\ \hline
        \textbf{calidad\_ingresos\_totales}    & Calidad de ingresos totales individuales \\ \hline
        \textbf{ingresos\_totales}             & Ingreso total individual percibido el mes anterior \\ \hline
        \textbf{calidad\_ingresos\_familiares} & Calidad de ingresos totales familiares \\ \hline
        \textbf{ingresos\_familiares}          & Ingresos totales familiares percibido el mes anterior \\ \hline
        \textbf{ing\_per\_cap\_familiar}       & Ingreso familiar per capita percibido el mes anterior \\ \hline
        \textbf{estado\_educativo}             & Asistencia (pasada o presente) o no a algún establecimiento educativo \\ \hline
        \textbf{sector\_educativo}             & Sector al que pertenece el establecimiento educativo a que asiste \\ \hline
        \textbf{nivel\_actual}                 & Nivel cursado al momento de la encuesta \\ \hline
        \textbf{nivel\_max\_educativo}         & Máximo nivel educativo que se cursó \\ \hline
        \textbf{años\_escolaridad}             & Años de escolaridad alcanzados \\ \hline
        \textbf{lugar\_nacimiento}             & Lugar de nacimiento de la persona encuestada \\ \hline
        \textbf{afiliacion\_salud}             & Afiliación de salud de la persona encuestada \\ \hline
        \textbf{hijos\_nacidos\_vivos}         & Tiene o tuvo hijos nacidos vivos \\ \hline
        \textbf{cant\_hijos\_nac\_vivos}       & Cantidad de hijos nacidos vivos \\ \hline
        \textbf{dominio}                       & ¿La vivienda se ubica en una villa de emergencia? \\ \hline
        \textbf{Target}                        & Nivel máximo educativo \\ \hline
    \end{tabular}
    \caption{Diccionarios de variables actualizadas}
    \label{dictionary}
    \end{table}
 
    \newpage

    Continuando con el análisis exploratorio de datos, se ha analizado la presencia de nulos en el dataset. Para visualizarlo se ha utilizado un gráfico de barras que incluye a todas las variables. 
   
    \begin{figure}[H]
        \centering
        \includegraphics[scale=0.25]{Imagenes/NullsDS.png}
        \caption{Nulos.}
        \label{nulls}
    \end{figure}
 
    A partir del gráfico de barras, se ha identificado que la variable target posee 1054 valores nulos. Es importante tener este dato presente al momento de correr un algoritmo de clasificación.

    Por otro lado, los nulos correspondientes a la variable "hijos nacidos vivos" se dan ya que los hijos se cuentan siempre a la madre y no al padre para no duplicar sus valores.

    \subsection{Análisis univariado}

    Durante el apartado de análisis univariado se han aunado esfuerzos para analizar las variables de manera aislada. El foco principal se ha puesto en definir las variables que aportan información sobre los individuos que forman parte del dataset.
        \subsubsection{Género y edad}
           
            Se comienza con un pantallazo general sobre las primeras cualidades de los datos, como muestra representativa para la EPH, sobre quiénes son los ciudadanos representados en el dataset.
           
            \begin{figure}[H]
                \centering
                \includegraphics[scale=0.45]{Imagenes/AUGenero.png}
                \caption{Análisis de genero y edad.}
                \label{AU genre and age}
            \end{figure}
            
            En la variable ``género'' los datos parecen equilibrados en ambas categorías. Para el caso de la variable ``edad'', la distribución se asemeja a la de una normal.
           
            \subsubsection{Comuna}
           
            Se ha continuado por evaluar la variable ``comuna''. En la misma se muestra la comuna de la Ciudad de Buenos Aires del entrevistado, de manera de tener una ubicación geográfica. Se ha considerado primordial revisar esta variable a fin de corroborar que exista un balanceo de datos de cantidad de entrevistados pos comuna.
           
            Para ello, se ha generado un mapa. En concreto se ha partido del mapa de comunas de la Ciudad de Buenos Aires y se han transformado las variables para fusionar el mapa con la base de datos de manera que coincidan.
           
            \begin{figure}  [H]
                \centering
                \includegraphics[scale=0.30]{Imagenes/AUComuna.png}  
                \caption{Encuestados por comuna.}
                \label{AU comuna}
            \end{figure}
 
            Al observar estos gráficos se determina que las comunas 1,4,7 y 8 tienen mayor cantidad de casos. Queda por verse si en posteriores análisis será necesario abordar esta diferencia para evitar sesgos. Para eso, será necesario tomar en cuenta el porcentaje de la población total de cada comuna.
           

            \subsubsection{Años de escolaridad}
           
            En este apartado, mediante un gráfico de barras, se han analizado los años de escolaridad alcanzados por los encuestados:
           
            \begin{figure}[H]
                \centering
                \includegraphics[scale=0.6]{Imagenes/AUAnosEsc.png}
                \caption{Años de escolaridad alcanzados.}
                \label{AU years of scholarship}
            \end{figure}
           
            A simple vista se observan tres ``picos'': en el valor mínimo, alrededor del 7.5 y alrededor del 12.5. Se ha inferido que estos tres casos corresponden a no tener estudios, solo haber transcurrido el primario y haber transcurrido hasta la educación secundaria, respectivamente.
           
            \subsubsection{Máximo nivel educativo (Target)}
            
            Asimismo, se ha definido analizar el balance de los datos propios del Target. Se presentan los gráficos correspondientes.
            
            \begin{figure}[H]
                \centering
                \includegraphics[scale=0.3]{Imagenes/AUTarget.png} 
                \caption{Distribución de datos del target.}
                \label{AU Target}
            \end{figure}
           
            Con respecto a la distribución de los datos del Target se determina que el nivel máximo educativo alcanzado con mayor frecuencia es el secundario completo, seguido por el primario. Adicionalmente, el nivel secundario y primario explican casi el 77\% de los datos, por lo tanto hay un gran desbalance en el Target. Esto se debe tener en cuenta al momento de alcanzar las conclusiones del proyecto.
           
           \subsubsection{Ingreso familiar per cápita}
           
            Finalmente, se ha definido evaluar una variable que pone como objeto de estudio al grupo familiar en vez del individuo, ingresos familiares. Al respecto, se ha armado una función para graficar y modificar con el nivel del filtrado de la variable y obtener un histograma que permita apreciar mejor la distribución de la variable sin tantos outliers:
           
            \begin{figure}[H]
                \centering
                \includegraphics[scale=0.325]{Imagenes/AUIngFam.png}
                \caption{Análisis ingreso familiar.}
                \label{AU Family income}
            \end{figure}
           
            % Y como hay muchos outliers que impiden ver la distribución correctamente, los quitamos de los gráficos:
           
            % \includegraphics[scale=0.4]{Graf5.png}
 
            % Y como se puede ver, la distribución de los ingresos familiares esta \textbf{sesgada}.

            % \subsubsection{Años de escolaridad}
           
            % Analizamos mediante un gráfico de barras los años de escolaridad alcanzados por los encuestados:
           
            % \begin{center}
            %     \includegraphics[scale=0.6]{Imagenes/AUAnosEsc.png}
            % \end{center}
           
            % A simple vista se observan tres ``picos'': en el valor mínimo, alrededor del 7.5 y alrededor del 12.5. Podemos inferir que estos tres casos corresponden a no tener estudios, solo haber transcurrido el primario y haber transcurrido hasta la educación secundaria, respectivamente.
           
            % \subsubsection{Máximo nivel educativo (Target)}
           
            % \begin{center}
            %     \includegraphics[scale=0.3]{Imagenes/AUTarget.png}    
            % \end{center}
           
            % Podemos observar que el nivel máximo educativo más alcanzado es el secundario completo, seguido por el primario. Contrario de lo que habíamos intuido anteriormente, el nivel superior quedó en tercer lugar. Adicionalmente, el nivel secundario y primario explican casi el 77\% de los datos.
             Como se puede ver en la figura, la distribución de los ingresos familiares se encuentra \textbf{sesgada}, existe un desbalanceo de datos.

    
    \newpage
    
    \subsection{Análisis bivariado}
        \subsubsection{Análisis de variables numéricas}
 
        Para comenzar el análisis bivariado del problema se han realizado diferentes heat maps para analizar, en primera instancia, la correlación entre las variables numéricas.        
       
        \begin{figure}[H]
                \centering
            \includegraphics[scale=0.65]{Imagenes/ABSpearman.png}
            \caption{Correlación entre variables por método de Spearman.}
            \label{AB Spearman Correlation}
        \end{figure}
 
        En primera instancia, no se observan fuertes correlaciones. Sin embargo, se puede apreciar que la variable ``años\_escolaridad'' correlaciona moderadamente bien con variables relacionadas al ingreso.
 
        En concreto, la principal correlación positiva es ``años\_escolaridad'' con ingreso familiar per cápita, lo cual hace sentido teórico.
        
        Por otra parte, al graficar en base a un threshold se puede observar que los años de escolaridad alcanzados por los entrevistados tienen una relación de $66\%$ con la variable ``ingresos\_totales''
        % Estaría bien solo utilizar el gráfico de la derecha.

        \begin{figure}[H]
            \centering
            \includegraphics[scale=0.38]{Imagenes/ABThreshold.png}
            \caption{Correlación entre variables mayor a 0,6(izq) y 0,5(der).}
            \label{AB Treshold Correlation}
        \end{figure}
 
        Por último corremos una tabla de correlación y filtramos las de valores más altos
 
        \begin{table}[H]
            \centering
            \begin{tabular}{|l|l|l|}
            \hline
                Variable 1 & Variable 2 & Correlación \\ \hline
                ingreso\_total\_lab & ingresos\_totales & 0.80 \\ \hline
                ingresos\_familiares & ing\_per\_cap\_familiar & 0.76 \\ \hline
                ingresos\_totales & ing\_per\_cap\_familiar & 0.62 \\ \hline
                ingresos\_totales & años\_escolaridad & 0.60 \\ \hline
                edad & ingreso\_total\_no\_lab & 0.57 \\ \hline
                ingreso\_total\_lab & años\_escolaridad & 0.54 \\ \hline
            \end{tabular}
            \caption{Tabla de correlaciones.}
            \label{AB Correlation Table}
        \end{table}
 
        A modo de conclusión del análisis preliminar bivariado, se puede identificar que:
        \begin{itemize}
            \item Existe una alta correlación entre las variables relacionadas al ingreso.
            \item Existe una alta correlación entre los ingresos y los años de escolaridad.
            \item Existe una relación positiva entre la edad y los ingresos totales.
        \end{itemize}
        
 \vspace{1cm}
 
            Finalmente, se ha analizado la relación entre los ingresos totales de cada hogar y los ingresos familiares por edad:
            \begin{figure}[H]
                \centering
                \includegraphics[scale=0.2275]{Imagenes/ABCompVarNum.png}
                \caption{Ingresos totales y familiares por edad.}
                \label{AB Income and age relationship}
            \end{figure}
 
            Se puede ver que desde los 30 años en adelante el ingreso total de la persona se corresponde con el ingreso familiar. Por ende suele haber un único ingreso fuerte por grupo familiar.
 
        \subsubsection{Comparación de variables categóricas con numéricas}
 
            Dentro de esta sección se han comparado variables numéricas con otras categóricas, como es el caso del Target. Se ha comenzado por establecer la relación entre el sexo y variables numéricas propias de las categorías de ingreso, características de los encuestados y educación.
 
            \begin{figure}[H]
                \centering
                \includegraphics[scale=0.2295]{Imagenes/ABCatVsNum1.png}
                \caption{Correlación entre sexo y variables numéricas.}
                \label{AB genre and numeric variables}
            \end{figure}
 
            Como se adelantó previamente, los encuestados varones figuran sin hijos ya que los mismos se asignan unicamente a la madre a fin de no duplicarlos en el dataset. Por otro lado, las distribuciones de la variable años de escolaridad son similares para varones y mujeres.
            Respecto a las variables relacionadas con el ingreso, se ha decidido quitar outliers de las correlaciones anteriores a fin de establecer relaciones claras.
 
            \begin{figure}[H]
                \centering
                \includegraphics[scale=0.425]{Imagenes/ABCatVsNum2.png}
                \caption{Correlación entre sexo y variables numéricas sin outliers.}
                \label{AB genre and numeric variables without outliers}
            \end{figure}
 
            Mediante el análisis de las figuras se determina que en el caso de los ingresos totales individuales los varones perciben mayores ingresos, sin embargo, la diferencia no es significativa. En el caso de los ingresos familiares la distribución se da de manera similar. 
 
 \vspace{1cm}

          Luego del análisis bivariado preliminar realizado para enriquecer la descripción del dataset, se ha decidido correlacionar al Target con los ingresos.
            \begin{figure}[H]
            \centering
                \includegraphics[scale=0.3]{Imagenes/ABCatVsNum3Bis.png}
                \caption{Correlación entre ingresos totales y Target}
                \label{AB Target and Total Incomes}
            \end{figure}
 
            A simple vista puede apreciarse que, para el nivel inicial, la remoción de outliers en otra categoría sigue siendo insuficiente para mostrar la distribución real de la variable. Por lo tanto se han analizado con mayor profundidad los valores de esta categoría:
 
            \begin{figure}[H]
            \centering
                \includegraphics[scale=0.8]{Imagenes/ABCatVsNum4.png}
                \caption{Ingresos totales para encuestados de nivel inicial}
                \label{AB Total Incomes for Initial level}
            \end{figure}
 
            Lógicamente, la enorme mayoría de los ingresos tienen el valor inicial de 0, puesto que incluye a personas que en ese momento estaban cursando su educación inicial, por lo que tenían entre 2 y 6 años.

            Entonces, ha resultado correspondiente analizar como se distribuyen los ingresos familiares con respecto al Target:
 

        \newpage
 
        \subsubsection{Variable numéricas con comuna}

            Para relacionar variables numéricas con la comuna, se ha analizado la interacción entre la frecuencia para cada categoría del Target y las comunas mediante heat maps: 
 
            \begin{figure}[H]
            \centering
                \includegraphics[scale=0.65]{Imagenes/ABCatVsNum6.png}
                \caption{Encuestados por nivel educativo y comuna}
                \label{AB education level and location}
            \end{figure}
 
            Se ha observado que en el sur de la ciudad hay  mayor cantidad de encuestados con niveles de inicial, primario y secundario completo, mientras que el norte (particularmente el barrio de Palermo) tiene mayor cantidad de personas con estudios superiores. En menor medida también las comunas del este (comúnmente llamado el ``centro de la ciudad'') destacan por la cantidad de encuestados con nivel superior.

\vspace{1cm}

            A su vez, se presenta otro heat map que reúne en un único gráfico el promedio de años de escolaridad por comuna. Este gráfico refuerza la relación presentada en la figura anterior.

            \begin{figure}[H]
            \centering
                \includegraphics[scale=0.65]{Imagenes/ABCatVsNum7.png}
                \caption{Años de escolaridad por comuna}
                \label{AB years of education and location}
            \end{figure}
 
            A partir de los últimos dos gráficos se ha establecido una clara división geográfica del nivel educativo:
            \begin{itemize}
                \item Las comunas del norte son las que tienen mayor nivel educativo.
                \item Las comunas del centro tienen niveles medios.
                \item Las comunas del sur (con las comuna 6 en el centro de la ciudad como outlier) y la comuna 1 en el este son las que tienen niveles más bajos.
            \end{itemize}

            Esta relación encontrada aporta gran información a la descripción del dataset en términos de relación entre el Target y la distribución geográfica d ela población.
    
    \newpage
    
    \subsection{Análisis multivariado}
 
        Durante esta sección se ha trabajado a fin de establecer relaciones entre más de 2 variables. En concreto, se ha buscado relacionar al Target con variables de las categorías de ingresos, educación, geolocalización, entre otras.
        En ese sentido, se ha comenzado por establecer la relación entre años de escolaridad, nivel máximo educativo (Target) y los ingresos totales.
 
        \begin{figure}[H]
            \centering
            \includegraphics[scale=0.65]{Imagenes/AMIngVsAnosEsc.png}
            \caption{Relación entre años de escolaridad, Target e ingresos totales}
            \label{AM Scholar Years, Taget and Total Income}
        \end{figure}
 
        Al visualizar se ha concluido que:
 
        \begin{itemize}
            \item Hasta los 6 años todos los casos llegan al nivel inicial.
            \item En dos años en que aparece el primario completo: 7 y 12 años. Se estima que se debe a la división entre los que comenzaron su educación en la primaria y los que comenzaron en el nivel inicial.
            \item A partir de los 12 años se observa un aumento consistente de los ingresos totales.
        \end{itemize}
        \vspace{1cm}

        A su vez, se ha buscado relacionar el Target, con los ingresos y el dominio.
 
        \begin{figure}[H]
            \centering
            \includegraphics[scale=0.255]{Imagenes/AMIngVSTargetBis.png}
            \caption{Relación entre dominio, Target e ingresos familiares}
            \label{AM Dominio, Taget and Familiar Income}
        \end{figure}

 
        En este punto se ha alcanzado una conclusión interesante: no importa el nivel máximo educativo, los casos que no provienen de villas de emergencia (dominio=``villas\_de\_emergencia'') obtienen en promedio ingresos más altos en todos los niveles educativos. El alcanzar estudios superiores no parece homogeneizar ambos conjuntos. Esto se puede observar en el segundo gráfico, ya que el violín naranja acumula mayor cantidad de casos hacia la derecha, en comparación con los violines azules que tienen una mayor distribución.
        
        \newpage

        Continuando con el análisis a nivel grupo familiar, se ha evaluado la relación entre ingresos familiares y comuna según el máximo nivel educativo alcanzado (Target):
        \begin{figure}[H]
            \centering
            \includegraphics[scale=0.6]{Imagenes/AMIngVsComuna.png}
            \caption{Relación entre comuna, Target e ingresos familiares}
            \label{AM Location, Taget and Familiar Income}
        \end{figure}
 
        En esta instancia se ha visualizado que a medida que avanza el nivel educativo máximo (Target) se atenúan levemente las diferencias de ingresos familiares entre comunas. Queda pendiente cruzar estos datos con la edad, para saber si el hecho de incluir a menores de edad está sesgando los valores para nivel inicial, primario y secundario.
 
\newpage


\section{Modelos analíticos}

    En esta sección se ha trabajado en pos de utilizar algoritmos que modelen los datos de manera correcta y así poder alcanzar los objetivos propuestos para el proyecto. En concreto, se ha buscado clasificar el Target a partir de las variables restantes del dataset utilizando machine learning. 
    No obstante, se ha comenzado por transformar algunas variables para poder trabajar con los algoritmos:
    \begin{itemize}
        \item Se ha rectegorizado la variables ``Target'' en variables numéricas, es decir, a cada nivel educativo le asignamos un valor numérico del 1 al 4:
        \begin{itemize}
            \item \textbf{inicial=} 1,
            \item \textbf{prim\_completo=} 2,
            \item \textbf{sec\_completo=} 3,
            \item \textbf{superior=} 4,
        \end{itemize}
        \item Se ha reagrupado la variable ``comuna'' por regiones para reducir la dimensionalidad (en norte, centro, sur y oeste),
        \item y por último se han renombrado algunas variables para que sean más cortas.
    \end{itemize}

    Como resultado han quedado las siguientes variables:
    
    \begin{table}[H]
        \centering
        \begin{tabular}{clll}
            \multicolumn{4}{l}{Tamaño del set de datos: 14319 entries, 0 to 14318} \\
            \multicolumn{4}{l}{Con un total 31 columnas: (total 33 columns):} \\
            \#  & Columnas & Tipo de Dato & Entradas No Vacías \\ \hline
            0   & id & object & 14319 \\ 
            1   & nhogar & object & 14319 \\ 
            2   & miembro & object & 14319 \\ 
            3   & comuna & object & 14319 \\ 
            4   & dominio & object & 14319 \\ 
            5   & edad & int64 & 14319 \\ 
            6   & sexo & object & 14319 \\ 
            7   & parentesco\_jefe & object & 14319 \\ 
            8   & situacion\_conyugal & object & 14318 \\ 
            9   & num\_miembro\_padre & object & 14319 \\ 
            10  & num\_miembro\_madre & object & 14319 \\ 
            11  & estado\_ocupacional & object & 14319 \\ 
            12  & cat\_ocupacional & object & 14319 \\ 
            13  & calidad\_ingresos\_lab & object & 14319 \\ 
            14  & ingreso\_total\_lab & int64 & 14319 \\ 
            15  & calidad\_ingresos\_no\_lab & object & 14319 \\ 
            16  & ingreso\_total\_no\_lab & int64 & 14319 \\ 
            17  & calidad\_ingresos\_totales & object & 14319 \\ 
            18  & ingresos\_totales & int64 & 14319 \\ 
            19  & calidad\_ingresos\_familiares & object & 14319 \\ 
            20  & ingresos\_familiares & int64 & 14319 \\ 
            21  & ing\_per\_cap\_familiar & int64 & 14319 \\ 
            22  & estado\_educativo & object & 14319 \\ 
            23  & sector\_educativo & object & 14316 \\ 
            24  & nivel\_actual & object & 14319 \\ 
            25  & nivel\_max\_educativo & object & 13265 \\ 
            26  & años\_escolaridad & float64 & 14257 \\ 
            27  & lugar\_nacimiento & object & 14318 \\ 
            28  & afiliacion\_salud & object & 14315 \\ 
            29  & hijos\_nacidos\_vivos & object & 6535 \\ 
            30  & cant\_hijos\_nac\_vivos & int64 & 14319 \\ 
            31  & Target & Int64 & 13223 \\ 
            32  & region & object & 14319 \\
            \multicolumn{4}{l}{Tipos de datos: Int64(1), float64(1), int64(7), object(24)} \\
            \multicolumn{4}{l}{Memoria usada: 3.6+ MB } \\
        \end{tabular}
        \caption{Tabla de variables modificadas}
        \label{final table of inputs and Target}
    \end{table}

    \subsection{Tratados de nulos}

        A la hora de eliminar los nulos del set de datos, se ha definido armar una función para tener una lista limpia de variables con nulos, se comparten los resultados obtenidos:

        \begin{table}[H]
            \begin{tabular}{lr}
                situacion\_conyugal & 1 \\ 
                lugar\_nacimiento & 1 \\ 
                sector\_educativo & 3 \\ 
                afiliacion\_salud & 4 \\ 
                años\_escolaridad & 62 \\ 
                nivel\_max\_educativo & 1054 \\ 
                Target & 1096 \\ 
                hijos\_nacidos\_vivos & 7784 \\ 
            \end{tabular}
        \end{table}

        Entonces, cada una de las variables se ha tratado de manera particular dependiendo los tipos de datos que aloja.
        
\vspace{1cm}

        En ese sentido, para eliminar los valores nulos de la variable ``años\_escolaridad'', la cual tiene datos numéricos, se han reemplazado con la mediana por comuna y sexo.
        
\vspace{1cm}

        Por otro lado, a los nulos de las variables categóricas:
        \begin{itemize}
            \item ``lugar\_nacimiento'',
            \item ``situacion\_conyugal'',
            \item ``afiliacion\_salud'',
            \item ``sector\_educativo''
            \item ``hijos\_nacidos\_vivos''
        \end{itemize}
        se han reemplazado con la moda.
        
 \vspace{1cm}    
 
        Finalmente, se ha eliminado la variable ``nivel\_max\_educativo'' ya que no es de interes. Asimismo, se han eliminado los nulos del Target y se ha pasado el tipo de dato de la misma a entero. 
        

    \subsection{Target}

        \subsubsection{Borrado de variables}
    
        Hay muchas variables que no se han considerado escenciales dado que no aportan información relevante en el algoritmo de clasificación. Precisamente, brindan información repetida. A continuación se comparten las categoría que se han descartado para correr el algoritmo: 



        \begin{table}[H]
            \centering
            \begin{tabular}{rl}
                \textbf{Variables}              & \textbf{Motivo de eliminación} \\ \hline
                id:                             & no suma información para la clasificación, \\
                nhogar:                         & no suma información para la clasificación, \\
                parentesco\_jefe:               & no suma información para la clasificación, \\
                miembro:                        & no suma información para la clasificación, \\
                num\_miembro\_padre:            & no suma información para la clasificación, \\
                num\_miembro\_madre:            & no suma información para la clasificación, \\
                cat\_ocupacional:               & brinda la misma información que estado\_ocupacional, \\
                calidad\_ingresos\_lab:         & brinda la misma información que ingreso\_total\_lab, \\
                calidad\_ingresos\_no\_lab:     & brinda la misma información que ingreso\_total\_no\_lab, \\
                calidad\_ingresos\_totales:     & brinda la misma información que ingresos\_totales, \\
                calidad\_ingresos\_familiares:  & brinda la misma información que ingreso\_familiares, \\
                estado\_educativo:              & no aporta información para la clasificación, \\
                nivel\_actual:                  & no aporta información para la clasificación, \\
                hijos\_nacidos\_vivos:          & brinda la misma información que cant\_hijos\_nac\_vivos, \\
                comuna:                         & variable ya abordada en la variable 'región'. \\
            \end{tabular}
        \caption{Variables eliminadas}
        \label{Deleted variables}
    \end{table}

 \vspace{1cm}
 
Y como resultado, se tiene el dataset listo para el procesamiento:

        \begin{table}[H]
            \centering
            \begin{tabular}{clll}
                \multicolumn{4}{l}{Tamaño del set de datos: 14319 entries, 0 to 14318} \\
                \multicolumn{4}{l}{Con un total 31 columnas: (total 18 columnas):} \\
                \#  & Columnas & Tipo de Dato & Entradas No Vacías \\ \hline
                0 & dominio & object & 13223 \\ 
                1 & edad & int64 & 13223 \\ 
                2 & sexo & object & 13223 \\ 
                3 & situacion\_conyugal & object & 13223 \\ 
                4 & estado\_ocupacional & object & 13223 \\ 
                5 & ingreso\_total\_lab & int64 & 13223 \\ 
                6 & ingreso\_total\_no\_lab & int64 & 13223 \\ 
                7 & ingresos\_totales & int64 & 13223 \\ 
                8 & ingresos\_familiares & int64 & 13223 \\ 
                9 & ing\_per\_cap\_familiar & int64 & 13223 \\ 
                10 & sector\_educativo & object & 13223 \\ 
                11 & años\_escolaridad & float64 & 13223 \\ 
                12 & lugar\_nacimiento & object & 13223 \\ 
                13 & afiliacion\_salud & object & 13223 \\ 
                14 & cant\_hijos\_nac\_vivos & int64 & 13223 \\ 
                15 & Target & int32 & 13223 \\ 
                16 & region & object & 13223 \\ 
                \multicolumn{4}{l}{Tipos de datos: Int64(1), float64(1), int64(7), object(8)} \\
                \multicolumn{4}{l}{Memoria usada: 2.0+ MB} 
            \end{tabular}
            \caption{Dataset listo para el procesamiento}
            \label{Final Dataset}
        \end{table}
    \subsection{Procesamiento}

        Para preparar los datos para el modelado se ha generado una función que:
        \begin{itemize}
            \item  Divide el dataframe en X\_train, y\_train, X\_test e y\_test, haciendo la división entre test y el train en un 30\% y un 70\% respectivamente, con una semilla especifica.
            \item  Procesa el X\_train y el X\_test con un pipeline generado previamente, el cual convierte las variable numéricas con el minmaxscaler y las categóricas con one hot encoding.
        \end{itemize}

        Una vez aplicada dicha función al dataframe, se ha tenido lista la partición (con la misma cantidad de columnas) del mismo en X\_train, y\_train, X\_test e y\_test.

    \subsection{Árbol de decisión}

        \subsubsection{Primer modelo}

            Como primera aproximación, se ha utilizado un árbol de clasificación usando con parámetros:
            \begin{itemize}
                \item random\_state = 50,
                \item max\_depth=8,
                \item criterion=`gini',
            \end{itemize}
            para saber como performa y mejorarlo a partir de ahí.

 \vspace{1cm}
 
            Como resultado se ha oservaddo que el Accuracy score para el test ha sido de: 0.940 y la matriz de confusión ha dado:
            \begin{table}[H]
                \centering
                \begin{tabular}{|l|c|c|c|c|}
                \hline
                    ~ & Predicc. Inicial & Predicc. Primario & Predicc. Secundario & Predicc. Superior \\ \hline
                    Inicial & 445 & 0 & 1 & 0 \\ \hline
                    Primario & 0 & 961 & 8 & 9 \\ \hline
                    Secundario & 0 & 19 & 1693 & 59 \\ \hline
                    Superior & 1 & 57 & 84 & 630 \\ \hline
                \end{tabular}
                \caption{Matriz de confusión primer modelo}
                \label{First model confusion matrix}
            \end{table}

            Por lo tanto, se han alcanzado las siguientes métricas:

            \begin{table}[H]
                \centering
                \begin{tabular}{rrrrr}
                    ~ & precision & recall & f1-score & support \\ 
                    & & & & \\
                    Inicial    & 1.00& 1.00 & 1.0 & 446 \\ 
                    Primario   & 0.93 & 0.99 & 0.95 & 978 \\ 
                    Secundario & 0.95 & 0.96 & 0.95 & 1771 \\ 
                    Superior   & 0.90 & 0.82 & 0.86 & 772 \\ 
                    & & & & \\
                    accuracy & & & 0.94 & 3967 \\ 
                    macro avg & 0.94 & 0.94 & 0.94 & 3967 \\ 
                    weighted avg & 0.94 & 0.94 & 0.94 & 3967 \\ 
                \end{tabular}
                \caption{Metricas primer modelo}
                \label{First model metrics}
            \end{table}

            A simple vista, se aprecia que el modelo performa muy bien, dado su accuracy. Cabe destacar que se observa una diferencia de 14 puntos en el f1-score del la categoría superior con respecto a la incial, mientras que para las otras dos categorías es solo de 5 puntos. Asimismo, se ha calculado su sesgo y su varianza:
            \begin{itemize}
                \item \textbf{Bias o sesgo:} 96.89\% lo cual indica poco error, es decir, un sesgo bajo,
                \item \textbf{Variance=Test\_Score - Bias:} 2.89\%, lo que indica que la varianza también es baja.
            \end{itemize}


            \subsubsection*{Resultados}

            Entonces, el modelo tiene una \textbf{buena relación} de sesgo y varianza.

            De aquí, ha sido necesario ver cuáles eran las variables más importantes para el armado del modelo. Esto permite volver el modelo más robusto, al quitar las mismas. se ha determinado que la variable ``años\_escolaridad'' tiene una importancia del 84\%, por mucho superior al resto de variables.

            Este hecho resulta contradictorio dado que, durante el análisis EDA, no se ha visto un resultado que indique tal relación, por otro lado, si se analiza la relación conceptual entre ambas variables, resulta evidente.

            Por lo tanto, se ha definido desarrollar un nuevo modelo sin esta variable. El principal motivo es que los años de escolaridad es un dato que puede constatarse de forma conjunta con el nivel máximo educativo, por lo que tiene sentido que si no se cuenta con la variable target, tampoco se tenga en consideración la variable de los años de escolaridad.

            Así, se ha creado un dataset nuevo (llamado ``df2'') sin la variable ``años\_escolaridad'', para volver a aplicar la función ``procesador'' para dividir nuevamente el mismo y generar nuevos modelos.

        \subsubsection{Segundo modelo}

            Esta vez al correr el modelo, utilizando el nuevo dataset (``df2''), se ha definido aplicar el ``DecisionTreeClassifier'' solo con los parámetros:
            \begin{itemize}
                \item random\_state = 50,
                \item cirterion=`entropy'.
            \end{itemize}

\vspace{1cm}
            Esto ha dado como resultado:  

            \begin{table}[!ht]
                \centering
                \begin{tabular}{rrrrr}
                    ~ & precision & recall & f1-score & suppo \\
                    & & & & \\
                    Inicial    & 0.83 & 0.79 & 0.81 & 446 \\
                    Primario   & 0.45 & 0.26 & 0.33 & 978 \\
                    Secundario & 0.56 & 0.66 & 0.60 & 1771 \\
                    Superior   & 0.39 & 0.45 & 0.42 & 772 \\
                    & & & & \\
                    accuracy & & & 0.53 & 3967 \\
                    macro avg & 0.56 & 0.59 & 0.54 & 3967 \\
                    weighted avg & 0.53 & 0.53 & 0.52 & 3967 \\
                \end{tabular}
                \caption{Metricas segundo modelo.}
                \label{Second model metrics}
            \end{table}

            Luego, se ha evaluado el sesgo y la varianza:
            \begin{itemize}
                \item \textbf{Bias o sesgo:} 99.78\% que indica poco error, es decir, un sesgo bajo,
                \item \textbf{Variance=Test\_Score - Bias:} 46.39\%, lo que indica un nivel de varianza alto,
            \end{itemize}
            
            Analizando lo observado, se ha determinado que este modelo hacía \textbf{OVERFITTING}, y tenía un rendimiento bastante pobre respecto al f1-score.

            En consecuencia, el árbol performa peor sin esta variable, aumentando, especialmente, la varianza. Por lo tanto se ha optado por probar con un grid search.
        
        \subsubsection{Gridsearch con CV}
            
            Entre la grilla de parámetros para el Gridsearch se han elegido los siguientes:
            \begin{itemize}
                \item max\_depth: range(5,11),
                \item max\_features: range(1,14),
                \item criterion: [`gini',`entropy',`log\_loss'];
            \end{itemize}
            como estimador el ``DecisionTreeClassifier'' con el random\_state=50, con el cross-validation =10 y usando todos los procesadores.
            
\vspace{1cm}

           Esto ha mostrado que el mejor árbol de decisión posible obtiene 0.642. Para alcanzarlo, el árbol debe tener una profundidad de  6, utilizar  10  variables y usar el método ``gini''.

            Por lo tanto, se ha entrenado el modelo bajo estos mismos parámetros y obtenido el siguiente reporte de clasificación:

            \begin{table}[H]
                \centering
                \begin{tabular}{rrrrr}
                    ~ & precision & recall & f1-score & suppo \\
                    & & & & \\
                    Inicial    & 0.99 & 0.74 & 0.85 & 446 \\
                    Primario   & 0.49 & 0.21 & 0.30 & 978 \\
                    Secundario & 0.55 & 0.87 & 0.69 & 1771 \\
                    Superior   & 0.78 & 0.41 & 0.54 & 772 \\
                    accuracy & & & 0.60 & 3967 \\
                    macro avg & 0.70 & 0.56 & 0.59 & 3967 \\
                    weighted avg & 0.63 & 0.60 & 0.57 & 3967 \\
                \end{tabular}
                \caption{Metricas segundo modelo mejorado.}
                \label{New Second model metrics}
            \end{table}

            Luego, se han investigado el sesgo y la varianza:
            \begin{itemize}
                \item \textbf{Bias o sesgo:} 65.27\% que indica que se tienen muchos errores, es decir, un bias alto,
                \item \textbf{Variance=Test\_Score - Bias:} 5.14\%, por el otro lado la varianza es baja.
            \end{itemize}

            Por lo cual, el modelo hizo \textbf{UNDERFITTING}, y no se han notado grandes mejoríaa en cuanto al f1-score.

            \subsubsection*{Resultados}

            El utilizar el grid search ha permitido mejorar bastante el modelo, el cual había perdido bastante accuracy al retirar los años de escolaridad. La métrica que más se ha podido mejorar con este método fue la varianza, que pasó de 44.97\% a 5.14\%, pero en contra parte el acuracy bajo de 100\% a 65.27\%.

    \subsection{Random Forest Classifier}
        
        En una instancia posterior, se ha trabajado con random forest con el propósito de analizar los resultados que arroja este algoritmo. Se ha comenzado teniendo en cuenta todo el dataset, incluido la variable con los años de escolaridad. 

        \subsubsection{Tercer modelo}

            Como tercer modelo con el Random Forest Classifier, se han elegido los siguientes parámetros:
            \begin{itemize}
                \item n\_estimators=200,
                \item max\_depth=15,
                \item random\_state=50,
                \item criterion=`gini'.
            \end{itemize}

            Se han obtenido los siguientes resultados en cuanto a las métricas:
            \begin{table}[H]
                \centering
                \begin{tabular}{rrrrr}
                    ~ & precision & recall & f1-score & suppo \\
                    & & & & \\
                    Inicial    & 1.00 & 0.96 & 0.98 & 446 \\
                    Primario   & 0.86 & 0.95 & 0.90 & 978 \\
                    Secundario & 0.91 & 0.93 & 0.9 & 1771 \\
                    Superior   & 0.92 & 0.76 & 0.83 & 772 \\
                    & & & & \\
                    accuracy & & & 0.91 & 3967 \\
                    macro avg & 0.92 & 0.90 & 0.91 & 3967 \\
                    weighted avg & 0.91 & 0.91 & 0.90 & 3967 \\
                \end{tabular}
                \caption{Metricas tercer modelo}
                \label{Thrid model metrics}
            \end{table}

             El random forest ha performado bastante bien, es decir, mucho mejor que los modelos anteriores:
             \begin{itemize}
                \item \textbf{Bias o sesgo:} 97.80\% indica que existen bastantes errores, es decir hay un sesgo alto,
                \item \textbf{Variance=Test\_Score - Bias:} 7.20\%, que indica que la varianza es baja.
             \end{itemize}

            \subsubsection*{Resultados}

             Se ha obtenido un buen modelo, que no tiene grandes diferencias al mirar el f1-score. De todas maneras, se han buscado cuales eran las variables más importantes. En ese sentido, se ha encontrado que la variable años de escolaridad redujo la enorme importancia (a un 43.58\%) que tenía en el random tree. Sin embargo, para poder comparar, sigue correspondiendo quitarla del modelo. Posteriormente, se ha probado un modelo alternativo sin la variable años de escolaridad. 

        \subsubsection{Cuarto modelo}

            En este cuarto caso se han elegido los siguientes parámetros:
            \begin{itemize}
                \item n\_estimators=200,
                \item max\_depth=10,
                \item random\_state=50,
                \item criterion=`gini'.
            \end{itemize}

            Se muestran los resultados de las medidas de desempeño:
            \begin{table}[H]
                \centering
                \begin{tabular}{rrrrr}
                    ~ & precision & recall & f1-score & suppo \\
                    & & & & \\
                    Inicial    & 0.95 & 0.78 & 0.86 & 446 \\
                    Primario   & 0.54 & 0.23 & 0.32 & 978 \\
                    Secundario & 0.56 & 0.88 & 0.69 & 1771 \\
                    Superior   & 0.80 & 0.40 & 0.53 & 772 \\
                    & & & & \\
                    accuracy & & & 0.62 & 3967 \\
                    macro avg & 0.71 & 0.57 & 0.60 & 3967 \\
                    weighted avg & 0.65 & 0.62 & 0.59 & 3967 \\
                \end{tabular}
                \caption{Metricas cuarto modelo}
                \label{Fourth model metrics}
            \end{table}

            Para este caso tambien se han buscado los valores del sesgo y la varianza:
            \begin{itemize}
                \item \textbf{Bias o sesgo:} 89.11\% que indica que se tienen pocos errores, es decir, que el sesgo es bastante bajo,
                \item \textbf{Variance=Test\_Score - Bias:} 27.4\%, esto indica que existe un valor alto en la varianza.
             \end{itemize}

            A modo de conclusión, respecto al tercer modelo, ha emperoado su accuracy pero está muy cercano al mejor modelo de Decision Tree, mientras que crece mucho la varianza a un 27.4\% (unos 20 puntos). 
            Al igual que con el el DesicionTree, se ha probado mejorándolo con grid search.

        \subsubsection{Gridsearch con CV}

            En la grilla de parámetros para el Gridsearch se han elegido los siguientes:
            \begin{itemize}
                \item max\_depth: [5,7,10,15,None],
                \item max\_features: [5,8,10,30,41],
                \item n\_estimators: [200,300,500],
                \item criterion: [`gini',`entropy',`log\_loss'];
            \end{itemize}
            como estimador el ``RandomTreeClassifier'' que se ha utilizado en el último modelo, con el cross-validation=10 y  usando todos los procesadores.

           Como parte de los resultado, el mejor random forest posible ha obtenido 0.668. 

            Y para eso, el árbol ha tenido una profundidad de 15, utilizado  10  variables,  300  estimadores y utilizado el método ``gini''.
            
            Entonces, se ha entrenado el modelo bajo estos mismos parámetros y se ha obtenido el siguiente reporte de clasificación:

            \begin{table}[!ht]
                \centering
                \begin{tabular}{rrrrr}
                    ~ & precision & recall & f1-score & support \\
                    & & & & \\
                    Inicial    & 0.95 & 0.79 & 0.86 & 446 \\
                    Primario   & 0.56 & 0.23 & 0.33 & 978 \\
                    Secundario & 0.56 & 0.89 & 0.69 & 1771 \\
                    Superior   & 0.80 & 0.40 & 0.54 & 772 \\
                    & & & & \\
                    accuracy & & & 0.62 & 3967 \\
                    macro avg & 0.72 & 0.58 & 0.60 & 3967 \\
                    weighted avg & 0.65 & 0.62 & 0.59 & 3967 \\
                \end{tabular}
                \caption{Metricas cuarto modelo mejorado.}
                \label{New fourth model metrics}
            \end{table}
            
            \begin{itemize}
                \item \textbf{Bias o sesgo:} 90.65\% que indica que hay pocos errores, es decir, el sesgo es bajo,
                \item \textbf{Variance=Test\_Score - Bias:} 28.54\% $\Rightarrow$, y la varianza también es alta. 
            \end{itemize} 

            Esto denota que este modelo está haciendo \textbf{OVERFITING} y continua desempeñandose mal al ver el f1-score.
            
            \subsubsection*{Resultados}
            
             Al utilizar random forest  se ha podido mejorar el sesgo y disminuir el underfitting en 2 puntos porcentuales aproximadamente.
    
            No obstante, se ha visto afectada la varianza en estos modelos, que pasó de estar alrededor del 5\% en el árbol de decisión mejorado a 28\%.

\newpage

\section{Conclusiones Finales}

    Finalmente, resta elegir el mejor modelo para realizar las predicciones. Para eso se han evaluado las métricas de cada uno de ellos, siempre teniendo en cuenta los modelos que no contienen la variable ``años\_escolaridad'', y hacer un cuadro comparativo:
    
    \begin{table}[H]
        \centering
        \begin{tabular}{|l|r|r|r|r|r|r|r|r|}
            \hline
           N°modelo & modelo & accuracy & sesgo & varianza & f1\_inicial & f1\_primario & f1\_secundario & f1\_superior \\ \hline
           2 & Árbol default & 0.53 & 1.00 & 0.46 & 0.81 & 0.33 & 0.60 & 0.42 \\ \hline
           2 &  Árbol mejorado & 0.60 & 0.65 & 0.05 & 0.85 & 0.30 & 0.69 & 0.53 \\ \hline
           4 & Bosque default & 0.62 & 0.89 & 0.27 & 0.86 & 0.32 & 0.69 & 0.53 \\ \hline
           4 & Bosque mejorado & 0.62 & 0.91 & 0.29 & 0.86 & 0.33 & 0.69 & 0.54 \\ \hline
        \end{tabular}
        \caption{Tabla general de resultados}
        \label{Result table}
    \end{table}

    Con esta información se ha definido qué modelo nos conviene usar:
    \begin{itemize}
        \item El árbol default tiene el mejor resultado con respecto al sesgo, pero su varianza lo deja afuera de consideración.
        \item Por el contrario, el árbol mejorado tiene una varianza insuperable de 5\%, aunque con el menor puntaje con respecto al sesgo.
        \item El bosque default tiene resultados mixtos en ambas categorías.
        \item El bosque mejorado destaca por bajo sesgo pero su varianza es la segunda más alta.
    \end{itemize}
    
    Se considera que son los finalistas son el árbol mejorado y el bosque mejorado. Llamativamente, ambos performan muy bien pero en métricas diferentes. A su vez, el accuracy de ambos difiere en apenas un 2\%. Por un lado, ambos modelos tienen un F1 Score similar, de 85\% y 86\% respectivamente. Sin embargo, el arbol mejorado presenta signos de underfitting cuando se lo compara con el otro modelo. A la inversa, el bosque mejorado muestra un escenario de overfitting con mayor varianza en la predicción. Cabe mencionar que aunque el bosque default tiene el mismo valor de F1 Score que el bosque mejorados, posee una probabilidad levemente mayor de underfitting. 

\vspace{1cm}
    
    En conclusión, en nuestra opinión, la mejor elección de modelo es el árbol mejorado, ya que tiene la mayor robustez  para poder generalizar en caso de agregar nuevos datos al modelo que difieran en gran medida con los existentes dentro del modelo, a comparación de los otros modelos, ya que presenta muy bajo overfitting. Otra ventaja frente al bosque aleatorio es su mayor velocidad de entrenamiento, asì como su capacidad de ser visualizada en un gráfico.

\vspace{1cm}

    En futuras líneas de investigación se debería investigar en profundidad el desbalanceo de datos propio del Target. Como se ha visto en la etapa de EDA, el Target tiene un importante desequilibrio en el volumen de datos de cada categoría. De continuar con el proyecto, se indagaría acerca de dicho desbalance y se mitigaría la problemática agregando datos en categorías con deficit y eliminando datos de categorías en exceso

    
    
\end{document}
