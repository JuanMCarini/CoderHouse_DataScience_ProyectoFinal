\documentclass[a4paper]{article}

\usepackage[utf8]{inputenc}
\usepackage[spanish]{babel}

\usepackage[top=2cm, bottom=1.5cm, right=1.5cm, left=2cm]{geometry}

\usepackage{graphicx}

\usepackage{hyperref}%Links
\hypersetup{  %formato link
    colorlinks=true,
    linkcolor=blue,
    filecolor=blue,      
    urlcolor=blue}

\usepackage{float} %para fijar tablas
\usepackage{multirow} %Para tablas

\title{\textbf{CoderHouse \\ \vspace{0.5cm} Curso Data Science \\ \vspace{0.5cm} Informe del Proyecto Final\\ \vspace{0.5cm}
Análisis socioeducativo de los habitantes de la Ciudad de Buenos Aires}}

\author{\textbf{Profesor:} Damian Dapueto \\ \vspace{0.2cm} \textbf{Tutor:} Héctor Alonso \\ \vspace{0.2cm} \textbf{Grupo de Trabajo:} \\ Lucia Buzzeo, Lucia Hukovsky,\\ Jose Saint German, Juan Martín Carini}

\begin{document}

\maketitle

\begin{center}
    \includegraphics[scale=1.75]{Imagenes/Coder.png}
\end{center}

\thispagestyle{empty}

\newpage

\tableofcontents

\newpage

\section{Presentación del problema y fuente de información}

    \subsection{Presentación del problema}

        Nos es de gran de interés vivir en una comunidad con políticas públicas eficaces que mejoren las condiciones de vida de las personas. En este sentido, hemos decidido analizar los diferentes ejes que en nuestro país se rigen por políticas publicas. Al respecto, encontramos una gran limitación en el eje de educación al reconocer que su acceso dista de ser equitativo. Este aspecto no nos resultó una novedad, sin embargo, nos dio el pie para comenzar una investigación que permita dar una explicación teórica a la problemática. 
        En concreto, nos ha permitido conocer mejor la situación educativa actual de CABA y descubrir las principales variables que afectan el nivel educativo.
        
        El análisis realizado en el marco del presente proyecto podría establecer una base de requerimientos que permitan generar políticas públicas efectivas, no solo en el ámbito educativo, sino en el económico, cultural, social y geográfico, entre otros.

    \subsection{Definición de la fuente de información}

        Para trabajar esta problemática, hemos decidido recurrir a la \href{https://www.estadisticaciudad.gob.ar/eyc/?page_id=702}{datos abiertos} del Gobierno de la Ciudad de Buenos Aires para el año 2019. El mismo está disponible en la base de \href{https://data.buenosaires.gob.ar/dataset/encuesta-anual-hogares/resource/3a45c563-396d-42de-ba93-8a93729e0723}{Encuesta Anual de Hogares} del GCBA.

        Esta encuesta contiene información demográfica, social, económica, educativa y de salud de 14319 habitantes de la Ciudad, la cual es una muestra representativa que permite obtener un vistazo de la población de la Ciudad.

\section{Pregunta y objetivos de investigación}

    Nuestro objetivo principal es descubrir las principales variables intervinientes en el nivel máximo educativo alcanzado por la población de la Ciudad Autónoma de Buenos Aires (CABA).

    De este objetivo principal se desprenden los siguientes sub-objetivos:
    \begin{itemize}
        \item Determinar si la ubicación geográfica del encuestado es determinante para alcanzar ciertos niveles educativos. De este objetivo se desprende determinar la relación entre el nivel educativo y la comuna del encuestado, así como la relación entre la misma variable y el hecho de que el encuestado habite en una villa de emergencia.
        \item Establecer la fuerza con la que el nivel socio-económico afecta la variable target.
        \item Explorar la relación del target con otras variables, como el sexo del encuestado, la cantidad de hijos, la afiliación de salud o la edad.
    \end{itemize}

\section{Orden de trabajo}

    Este trabajo estará divido en 3 partes:
    \begin{enumerate}
        \item \textbf{Introducción a las variables del problema:} Se hará un análisis de las variables en donde se buscará conocer su performance dentro del dataset y su potencial signifícanos para la pregunta que buscamos responder. A la vez, queremos ver cómo las variables interactúan entre si. Esta parte es lo que se conoce como análisis univariado, bivariado y multivariado,
        \item \textbf{Modelos analíticos:} En esta sección se llevarán a cabo diversos modelos analíticos y algoritmos que nos servirán para acercarnos a la respuesta a nuestra pregunta de investigación,
        \item \textbf{Conclusión:} Haremos conclusiones finales sobre nuestros hallazgos. Además, discutiremos posibles limitaciones que tuviera y plantearemos futuras líneas de análisis a partir del análisis presente.
    \end{enumerate}

\section{Introducción a las variables: Análisis exploratorio de los datos}

    Una ves cargado el dataset con el que vamos a trabajar, miramos sus variable, el tipo que son y si tienen nulls:
    \begin{table}[H]\begin{center}
    \begin{tabular}{clll}
    \multicolumn{4}{l}{RangeIndex: 14319 entries, 0 to 14318} \\
    \multicolumn{4}{l}{Data columns (total 31 columns):}  \\
    \#  & Column                     & Non-Null Count & Dtype \\ \hline
    0  & id                          & 14319 non-null & int64 \\ 
    1  & nhogar                      & 14319 non-null & int64 \\ 
    2  & miembro                     & 14319 non-null & int64 \\ 
    3  & comuna                      & 14319 non-null & int64 \\ 
    4  & dominio                     & 14319 non-null & object \\
    5  & edad                        & 14319 non-null & int64 \\ 
    6  & sexo                        & 14319 non-null & object \\
    7  & parentesco\_jefe             & 14319 non-null & object \\
    8  & situacion\_conyugal          & 14318 non-null & object \\
    9  & num\_miembro\_padre           & 14319 non-null & object \\
    10 & num\_miembro\_madre           & 14319 non-null & object \\
    11 & estado\_ocupacional          & 14319 non-null & object \\
    12 & cat\_ocupacional             & 14319 non-null & object \\
    13 & calidad\_ingresos\_lab        & 14319 non-null & object \\
    14 & ingreso\_total\_lab           & 14319 non-null & int64  \\
    15 & calidad\_ingresos\_no\_lab     & 14319 non-null & object \\
    16 & ingreso\_total\_no\_lab        & 14319 non-null & int64  \\
    17 & calidad\_ingresos\_totales    & 14319 non-null & object \\
    18 & ingresos\_totales            & 14319 non-null & int64  \\
    19 & calidad\_ingresos\_familiares & 14319 non-null & object \\
    20 & ingresos\_familiares         & 14319 non-null & int64  \\
    21 & ingreso\_per\_capita\_familiar & 14319 non-null & int64  \\
    22 & estado\_educativo            & 14319 non-null & object \\
    23 & sector\_educativo            & 14316 non-null & object \\
    24 & nivel\_actual                & 14319 non-null & object \\
    25 & nivel\_max\_educativo         & 13265 non-null & object \\
    26 & años\_escolaridad            & 14257 non-null & object \\
    27 & lugar\_nacimiento            & 14318 non-null & object \\
    28 & afiliacion\_salud            & 14315 non-null & object \\
    29 & hijos\_nacidos\_vivos         & 6535 non-null  & object \\
    30 & cantidad\_hijos\_nac\_vivos    & 14319 non-null & object \\
    \multicolumn{4}{l}{dtypes: int64(10), object(21)}  \\ 
    \multicolumn{4}{l}{memory usage: 3.4+ MB} \\
    \end{tabular}\end{center}
    \end{table}
        
    Luego, generamos diversas transformaciones de variables, así como la creación de la variable "Target", pues es la que usaremos para todo el análisis:
    \begin{itemize}
        \item creamos el target para nivel\_max\_educativo,
        \item remplazamos los valores de años\_escolaridad para que todos sean numéricos,
        \item la variable ``cantidad\_hijos\_nac\_vivos'' se puede pasar a numérica si se toma ``no corresponde'' como NAN,
        \item hay determinadas variables (comuna, id, hogar y miembro) que están como numéricas pero deberían ser categóricas,
        \item por otro lado, variables como sexo y dominio pueden pasarse a numérico mediante one hot encoding,
        \item generamos la variable ``target'' como copia de ``Target'' para tener ambas versiones,
        \item pasamos la variable Target a one hot encoding,
        \item por último renombramos algunas variables para que sean más cortas.
    \end{itemize}

    Ahora, con el dataset ya acomodado, comenzamos analizándolo en su conjunto.    Miramos las nuevas modificaciones en las variable, el tipo que son y si tienen nulls:
    \begin{table}[H]\begin{center}
    \begin{tabular}{clll}
        \multicolumn{4}{l}{RangeIndex: 14319 entries, 0 to 14318} \\
        \multicolumn{4}{l}{Data columns (total 36 columns):} \\
        \#  & Column                     & Non-Null Count & Dtype \\ \hline
        0 & id                           & 14319 non-null & object \\
        1 & nhogar                       & 14319 non-null & object \\
        2 &  miembro                     & 14319 non-null & object \\ 
        3 &  comuna                      & 14319 non-null & object  \\
        4 &  edad                        & 14319 non-null & int64   \\
        5 &  parentesco\_jefe             & 14319 non-null & object  \\
        6 &  situacion\_conyugal          & 14318 non-null & object  \\
        7 &  num\_miembro\_padre           & 14319 non-null & object  \\
        8 &  num\_miembro\_madre           & 14319 non-null & object  \\
        9 &  estado\_ocupacional          & 14319 non-null & object  \\
        10 &  cat\_ocupacional            & 14319 non-null & object  \\
        11 & calidad\_ingresos\_lab        & 14319 non-null & object  \\
        12 & ingreso\_total\_lab           & 14319 non-null & int64   \\
        13 & calidad\_ingresos\_no\_lab     & 14319 non-null & object  \\
        14 & ingreso\_total\_no\_lab        & 14319 non-null & int64   \\
        15 & calidad\_ingresos\_totales    & 14319 non-null & object  \\
        16 & ingresos\_totales            & 14319 non-null & int64   \\
        17 & calidad\_ingresos\_familiares & 14319 non-null & object  \\
        18 & ingresos\_familiares         & 14319 non-null & int64   \\
        19 & ing\_per\_cap\_familiar        & 14319 non-null & int64   \\
        20 & estado\_educativo            & 14319 non-null & object  \\
        21 & sector\_educativo            & 14316 non-null & object  \\
        22 & nivel\_actual                & 14319 non-null & object  \\
        23 & nivel\_max\_educativo         & 13265 non-null & object  \\
        24 & años\_escolaridad            & 14257 non-null & float64 \\
        25 & lugar\_nacimiento            & 14318 non-null & object  \\
        26 & afiliacion\_salud            & 14315 non-null & object  \\
        27 & hijos\_nacidos\_vivos         & 6535 non-null  & object  \\
        28 & cant\_hijos\_nac\_vivos        & 14319 non-null & int64   \\
        29 & sexo\_Varon                  & 14319 non-null & uint8   \\
        30 & dominio\_villas              & 14319 non-null & uint8   \\
        31 & target                      & 13223 non-null & object  \\
        32 & Target\_inicial              & 14319 non-null & uint8   \\
        33 & Target\_prim\_completo        & 14319 non-null & uint8   \\
        34 & Target\_sec\_completo         & 14319 non-null & uint8   \\
        35 & Target\_superior             & 14319 non-null & uint8   \\
        \multicolumn{4}{l}{dtypes: float64(1), int64(7), object(22), uint8(6)}  \\
        \multicolumn{4}{l}{memory usage: 3.4+ MB}
    \end{tabular}\end{center}
    \end{table}

    \begin{center}
        \includegraphics[scale=0.25]{Imagenes/NullsDS.png}
    \end{center}

    Detectamos que nuestra variable target tiene 1054 valores nulos. Es importante tener este dato presente cuando querramos correr un algoritmo de clasificación.

    \subsection{Análisis univariado}

        \subsubsection{Género y edad}
            
            Comenzamos con un pantallazo general sobre las primeras cualidades de los datos, como muestra representativa para la EPH, sobre quiénes son los ciudadanos representado en el dataset.
            
            \begin{center}
                \includegraphics[scale=0.45]{Imagenes/AUGenero.png}    
            \end{center}
            

            En la variable género los datos parecen equilibrados en las categorías. Para el caso de la variable "edad", la distribución se asemeja a la de una normal.
            
            \subsubsection{Comuna}
            
            Seguimos observando la variable ``comuna". En la misma se muestra la comuna de la Ciudad de Buenos Aires del entrevistado, de manera de tener una ubicación geográfica. Consideramos importante revisar esta variable ya que tenemos como hipótesis que el nivel educativo alcanzado puede estar dependiendo de la zona geográfica de la ciudad en la que se encuentra el entrevistado.
            
            Para esto vamos a generar un mapa, así que utilizaremos el mapa de comunas de la Ciudad de Buenos Aires, transformamos las variables que vamos a usar para joinear el mapa con la base de manera que coincidan, transformamos la base para contablizar la frecuencia con la que aparece cada comuna en la base. Y por último unimos ambos datasets y generamos una nueva variable con las coordenadas para poder agregar etiquetas en el centro geográfico de cada comuna, que nos da como resultado los siguientes graficos:
            
            \begin{center}
                \includegraphics[scale=0.30]{Imagenes/AUComuna.png}    
            \end{center}

            Observando ambos gráficos vemos que las comunas 1,4,7 y 8 tienen mayor cantidad de casos. Queda por verse si en posteriores análisis es necesario abordar esta diferencia para evitar sesgos. Para eso, será necesario tomar en cuenta el porcentaje de la población total de cada comuna.
            
            \subsubsection{Ingreso familiar per capita}
            
            Ahora probamos con observar los ingresos familiares. Creemos que puede ser un indicador interesante del nivel educativo.
            
            Para esto, armamos una función para graficar y jugar con el nivel del filtrado de la variable y obtener un histograma que permita apreciar mejor la distribución de la variable sin tantos outliers. Probamos graficando con el máximo de la variable:
            
            \begin{center}
                \includegraphics[scale=0.325]{Imagenes/AUIngFam.png}
            \end{center}
            
            Y como hay muchos outliers que impiden ver la distribución correctamente, los quitamos de los gráficos:
            
            % \includegraphics[scale=0.4]{Graf5.png}

            De este forma vemos que, aún removiendo los outliers, la distribución sigue sesgada.

            \subsubsection{Años de escolaridad}
            
            Analizamos mediante un gráfico de barras los años de escolaridad alcanzados por los encuestados:
            
            \begin{center}
                \includegraphics[scale=0.6]{Imagenes/AUAnosEsc.png}
            \end{center}
            
            A simple vista se observan tres "picos": en el valor mínimo, alrededor del 7.5 y alrededor del 12.5. Podemos inferir que estos tres casos corresponden a no tener estudios, solo haber transcurrido el primario y haber transcurrido hasta la educación secundaria, respectivamente.
            
            \subsubsection{Máximo nivel educativo (Target)}
            
            \begin{center}
                \includegraphics[scale=0.3]{Imagenes/AUTarget.png}    
            \end{center}
            
            Podemos observar que el nivel máximo educativo más alcanzado es el secundario completo, seguido por el primario. Contrario de lo que habíamos intuido anteriormente, el nivel superior quedó en tercer lugar. Adicionalmente, el nivel secundario y primario explican casi el 77\% de los datos.

    \subsection{Análisis bivariado}

    \subsection{Análisis multivariado}

\section{Modelos analíticos}

\end{document}